\documentclass[12pt]{article}

\usepackage[T1]{fontenc}
\usepackage[utf8]{inputenc}
\usepackage[russian]{babel}

% page margin
\usepackage[top=2cm, bottom=2cm, left=2cm, right=2cm]{geometry}

\usepackage{graphicx}

% AMS packages
\usepackage{amsmath}
\usepackage{amssymb}
\usepackage{amsfonts}
\usepackage{amsthm}

% blackboar lettering
\usepackage{dsfont}
\usepackage{bbm}

\usepackage{fancyhdr}
\pagestyle{fancy}
% modifying page layout using fancyhdr
\fancyhf{}
\renewcommand{\sectionmark}[1]{\markright{\thesection\ #1}}
\renewcommand{\subsectionmark}[1]{\markright{\thesubsection\ #1}}

\rhead{\fancyplain{}{\rightmark }}
\cfoot{\fancyplain{}{\thepage }}

\usepackage{titlesec}

% for appendix environment
\usepackage[titletoc,toc,title,page]{appendix}

\newcommand{\lb}{\left(}
\newcommand{\rb}{\right)}

\newcommand{\mL}{\mathcal{L}}
\newcommand{\mH}{\mathcal{H}}

\begin{document}

Попытаюсь сохранить обозначения, которые вы использовали в последней имеющейся у меня версии текста по поводу якобианов. \par
Исходно кинетическая энергия $T_\mL$ представлена в виде квадратичной формы
\begin{gather}
	T_\mL = \frac{1}{2} \boldsymbol{v}^\dagger \Pi \boldsymbol{v} \notag
\end{gather}

Кинетическая энергия $T_\mH$, соответствующая $T_\mL$, также может записана как квадратичная форма
\begin{gather}
	T_\mH = \frac{1}{2} \boldsymbol{p}^\dagger \Pi^{-1} \boldsymbol{p} \notag
\end{gather}

Вы рассматриваете неособенное линейное преобразование $\boldsymbol{u} = \mathbb{K} \boldsymbol{v}$, приводящее квадратичную форму $T_\mL$ к диагональному виду
\begin{gather}
	T_\mL = \frac{1}{2} \boldsymbol{u}^\dagger \Pi_d \boldsymbol{u} = \frac{1}{2} \boldsymbol{u}^\dagger \lb \mathbb{K}^{-1} \rb^\dagger \Pi \mathbb{K}^{-1} \boldsymbol{u} \notag
\end{gather}

Квадратичная форма $T_\mH$, порожденная квадратичной формой $T_\mL$ также диагональна 
\begin{gather}
		T_\mH = \frac{1}{2} \boldsymbol{\Pi}^\dagger \Pi_d^{-1} \boldsymbol{\Pi}, \notag
\end{gather}
где
\begin{gather}
	\boldsymbol{\Pi} = \Pi_d \boldsymbol{u} = \lb \mathbb{K}^{-1} \rb^\dagger \boldsymbol{p}. \notag
\end{gather}

Но нас ведь интересуют якобианы, которые приводят квадратичные формы к суммы совершенно голых квадратов. Осуществим переход к голым квадратам через промежуточный этап с голыми квадратами с одной второй перед ними. Преобразование переменных $\boldsymbol{\omega} = \Pi_d^{-\frac{1}{2}} \boldsymbol{p}$ приводит квадратичную форму $T_\mL$ к виду
\begin{gather}
	T_\mL = \frac{1}{2} \boldsymbol{\omega}^\dagger \boldsymbol{\omega} \notag
\end{gather}
Квадратичная форма $T_\mH$ приходит к аналогичному виду
\begin{gather}
	T_\mH = \frac{1}{2} \boldsymbol{\zeta}^\dagger \boldsymbol{\zeta},  \notag
\end{gather}
где
\begin{gather}
	\boldsymbol{\zeta} = \Pi_d^\frac{1}{2} \boldsymbol{\Pi} \notag
\end{gather}

Наконец, преобразование $\boldsymbol{\xi} = \frac{1}{\sqrt{2}} \boldsymbol{\omega}$ приводит квадратичную форму $T_\mL$ к голым квадратам
\begin{gather}
	T_\mL = \boldsymbol{\xi}^\dagger \boldsymbol{\xi} \notag
\end{gather}

Аналогичное преобразование $\boldsymbol{\eta} = \frac{1}{\sqrt{2}} \boldsymbol{\zeta}$ приводит квадратичную форму $T_\mH$ к голым квадратам
\begin{gather}
	T_\mH = \boldsymbol{\eta}^\dagger \boldsymbol{\eta} \notag
\end{gather}

Итак, рассмотрим якобианы полных преобразований переменных, приводящих квадратичные формы $T_\mL$ и $T_\mH$ к голым квадратам, в виде произведения трех якобианов частичных преобразований:
\begin{gather}
	Yac_\mL = \det \mathbb{K}^{-1} \cdot \lb \det \Pi_d \rb ^{-\frac{1}{2}} \cdot 2^\frac{s}{2} \notag \\
	Yac_\mH = \det \mathbb{K} \cdot \lb \det \Pi_d \rb^\frac{1}{2} \cdot 2^\frac{s}{2}, \notag
\end{gather}

где через $s$ было обозначено количество квадратов. Их произведение дает
\begin{gather}
	Yac_\mL \cdot Yac_\mH = 2^\frac{s}{2} \cdot 2^\frac{s}{2} = 2^s. \notag
\end{gather}


\end{document}
