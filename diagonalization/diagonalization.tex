\documentclass[12pt]{article}

\usepackage[T1]{fontenc}
\usepackage[utf8]{inputenc}
\usepackage[russian]{babel}

% page margin
\usepackage[top=2cm, bottom=2cm, left=2cm, right=2cm]{geometry}

\usepackage{graphicx}

% AMS packages
\usepackage{amsmath}
\usepackage{amssymb}
\usepackage{amsfonts}
\usepackage{amsthm}

% blackboar lettering
\usepackage{dsfont}
\usepackage{bbm}

\usepackage{fancyhdr}
\pagestyle{fancy}
% modifying page layout using fancyhdr
\fancyhf{}
\renewcommand{\sectionmark}[1]{\markright{\thesection\ #1}}
\renewcommand{\subsectionmark}[1]{\markright{\thesubsection\ #1}}

\rhead{\fancyplain{}{\rightmark }}
\cfoot{\fancyplain{}{\thepage }}

\usepackage{titlesec}

% for appendix environment
\usepackage[titletoc,toc,title,page]{appendix}

\newcommand{\lb}{\left(}
\newcommand{\rb}{\right)}

\newcommand{\mL}{\mathcal{L}}
\newcommand{\mH}{\mathcal{H}}
\newcommand{\mN}{\mathcal{N}}
\newcommand{\intty}{\int\limits_{-\infty}^{+\infty}}

\usepackage{dsfont}
\usepackage{bbm}

\newcommand{\bbB}{\mathbb{B}}
\newcommand{\bbU}{\mathbb{U}}
\newcommand{\bbG}{\mathbb{G}}
\newcommand{\bbE}{\mathbb{E}}
\newcommand{\bbS}{\mathbb{S}}
\newcommand{\bbV}{\mathbb{V}}
\newcommand{\bfj}{\mathbf{j}}
\newcommand{\bfJ}{\mathbf{J}}

\newcommand{\mf}{\mathbf}

\newcommand{\dR}{\dot{R}}
\newcommand{\dT}{\dot{\Theta}}
\newcommand{\pT}{p_\Theta}

\newcommand{\bbI}{\mathds{I}}
\newcommand{\bba}{\mathbbm{a}}
\newcommand{\bbA}{\mathds{A}}

\newcommand{\JacL}{\left[ \, Jac \, \right]_{lagr}}
\newcommand{\JacH}{\left[ \, Jac \, \right]_{ham}}

\newcommand{\dtone}{\dot{\Theta}_1}
\newcommand{\dttwo}{\dot{\Theta}_2}
\newcommand{\dphi}{\dot{\varphi}}

\newcommand{\ptone}{p_{\Theta_1}}
\newcommand{\pttwo}{p_{\Theta_2}}
\newcommand{\pphi}{p_{\varphi}}

\usepackage{algorithm}
\usepackage[noend]{algpseudocode}

\titleformat{\section}{\bfseries}{\thesection.}{1em}{}
\titleformat{\subsection}{\normalfont\itshape\bfseries}{\thesubsection.}{0.5em}{}

% tikz packages
\usepackage{tikz}
\usetikzlibrary{shapes.geometric, arrows, positioning, decorations.markings}
\usetikzlibrary{fit}
\usepackage{microtype}
\usepackage{framed}
\usetikzlibrary{decorations.pathmorphing,calc,backgrounds}


\title{О диагонализации гамильтониана}
\date{}

\begin{document}

\maketitle

Рассмотрим набор систем по возрастанию количества внутренних переменных. Для каждой из них выпишем лагранжиан в некоторой молекулярной системе отсчета и найдем такое преобразование переменных, которое диагонализует лагранжиан. Используем найденное преобразование для диагонализации гамильтониана. Затем выпишем якобианы замены переменных в лагранжиане и в гамильтониане.

\section{Двухатомная система}
Пусть совокупное движение двух тел происходит в плоскости $Oyz$. Выберем молекулярную систему отсчета таким образом, чтобы рассматриваемые тела находились на оси $Z$. В таком случае, плоскости молекулярной системы $OYZ$ и лабораторной системы $Oyz$ совпадают в любой момент времени. При этом угол между осями $Oy$ и $OY$ (равный, конечно, углу между $Oz$ и $OZ$) равен эйлеровому углу $\theta$ (в рамках стандартного определения эйлеровых углов по Голдстейну). Остальные два эйлеровых угла, $\phi$ и $\psi$, равны 0.

Понятно, что угловая скорость, соответсвующая повороту на угол $\theta$, направлена вдоль оси вращения $x$ = $X$. Следовательно, $\Omega_x = \dot{\theta}$. Аналогичный результат можно получить, взяв матрицу $\bbV$, связывающую компоненты угловой скорости с эйлеровыми скоростями, в общем случае и подставив в нее $\phi = \psi = 0$:
\begin{gather}
	\begin{bmatrix}
	\Omega_x \\
	\Omega_y \\
	\Omega_z 
	\end{bmatrix} = \bbV 
	\begin{bmatrix}
		\dot{\varphi} \\
		\dot{\theta} \\
		\dot{\psi}
	\end{bmatrix} \notag \\
	\bbV = \begin{bmatrix}
		\sin \theta \sin \psi & \cos \psi & 0 \\
		\sin \theta \cos \psi & - \sin \psi & 0 \\
		\cos \theta & 0 & 1
	\end{bmatrix} \rightarrow
	\begin{bmatrix}
		0 & 1 & 0 \\
		\sin \theta & 0 & 0 \\
		\cos \theta & 0 & 1
	\end{bmatrix} \notag
\end{gather}

Т.к. $\dot{\varphi} = \dot{\psi} = 0$, то $\Omega_x = \dot{\theta}$, $\Omega_y = 0$, $\Omega_z = 0$.


Выведем кинетическую энергию в лагранжевой форме в предложенной системе координат. Обозначим массы тел за $m_1$ и $m_2$, расстояние между телами -- за $R$. Тогда координаты тел равны
\begin{gather}
	\begin{aligned}
			x_1 &= 0 \\
			y_1 &= 0 \\
			z_1 &= - \frac{m_2}{m_1 + m_2} R
	\end{aligned}
	\quad \quad \quad
	\begin{aligned}
			x_2 &= 0 \\
			y_2 &= 0 \\
			z_2 &= \frac{m_1}{m_1 + m_2} R 
	\end{aligned} \notag
\end{gather}

Тензор инерции будет иметь лишь две ненулевые компоненты, а именно, $I_{xx} = I_{yy} = \mu R^2$, где $\mu$ -- приведенная масса системы, $\mu = \displaystyle \frac{m_1 m_2}{m_1 + m_2}$. Приведенные выше рассуждения показывают, что вектор угловой скорости направлен вдоль оси $x = X$:
\begin{gather}
	\mf{\Omega} = \begin{bmatrix}
		\Omega_x \\
		0 \\ 
		0
	\end{bmatrix} \notag
\end{gather}

Итак, кинетическая энергия в лагранжевой форме в рассматриваемой молекулярной системе отсчета имеет следующий вид
\begin{gather}
		T_\mL = \frac{1}{2} \mu \dot{R}^2 + \frac{1}{2} \mu R^2 \Omega_x^2 \notag 
\end{gather}

Осуществим следующую тривиальную замену переменных
\begin{gather}
	\begin{aligned}
		x_1^2 &= \frac{1}{2} \mu \dR^2 \\
		x_2^2 &= \frac{1}{2} \mu R^2 \Omega_x^2
	\end{aligned}, \label{diatom1}
\end{gather}

приводящую лагранжиан к сумме квадратов
\begin{gather}
		T_\mL = x_1^2 + x_2^2. \label{diatom2}
\end{gather}

Осуществим стандартную процедуру перехода от лагранжевой кинетической энергии к гамильтоновой, однако будем выполнять промежуточные преобразования через переменные $x_1$ и $x_2$. Выразим лагранжевы переменные $\dR$, $\Omega_x$ и гамильтоновые переменные $p_R$, $J_x$ через $x_1$ и $x_2$. 
\begin{gather}
	\begin{aligned}
			\dR &= \sqrt{\frac{2}{\mu}} x_1 \\
			\Omega_x &= \sqrt{\frac{2}{\mu R^2}} x_2 \\
	\end{aligned}
	\quad \quad \quad
	\begin{aligned}
			p_R &= \frac{\partial T_\mL}{\partial \dR} = \mu \dR = \sqrt{2 \mu} x_1 \\
			J_x &= \frac{\partial T_\mL}{\partial \Omega_x} = \mu R^2 \Omega_x = \sqrt{2 \mu R^2} x_2
	\end{aligned} \notag
\end{gather}

Получим выражение для гамильтоновой формы кинетической энергии
\begin{gather}
	T_\mH = \dR p_R + \Omega_x J_x - x_1^2 - x_2^2 \notag \\
	\begin{aligned}
			\dR \cdot p_R &= \sqrt{\frac{\mu}{2}} x_1  \cdot \sqrt{2 \mu} x_1 = 2 x_1^2 \\
			\Omega_x \cdot J_x &= \sqrt{\frac{2}{\mu R^2}} x_2 \cdot \sqrt{2 \mu R^2} x_2 = 2 x_2^2
	\end{aligned} \notag \\
	T_\mH = 2 x_1^2 + 2 x_2^2 - x_1^2 - x_2^2 = x_1^2 + x_2^2 \notag
\end{gather}

Итак, кинетическая энергия в гамильтоновой форме выражается в виде суммы квадратов таким же образом, как и кинетическая энергия в лагранжевой форме. Однако в выражениях переменных $x_1$ и $x_2$ необходимо лагранжевые переменные преобразовать в гамильтоновы переменные:
\begin{gather}
	x_1^2 = \frac{1}{2} \mu \dR^2 \quad \longrightarrow \quad x_1^2 = \frac{p_R^2}{2 \mu} \notag \\
	x_2^2 = \frac{1}{2} \mu R^2 \Omega_x^2 \quad \longrightarrow \quad x_2^2 = \frac{J_x^2}{2 \mu R^2} \notag \\
	T_\mL = x_1^2 + x_2^2 \quad \longrightarrow \quad T_\mH = x_1^2 + x_2^2 \notag
\end{gather}

Получим выражения для якобианов замены переменных в лагранжиане и гамильтониане. 
\begin{gather}
		\left[ \, Jac \, \right]_{lagr} = \Bigg{|} \frac{\partial ( \dR, \Omega_x )}{\partial ( x_1, x_2 )} \Bigg{|}  =  
	\begin{vmatrix}
			\displaystyle \frac{\partial \dR}{\partial x_1} & \displaystyle \frac{\partial \dR}{\partial x_2} \\[2ex]
		\displaystyle \frac{\partial \Omega_x}{\partial x_1} & \displaystyle \frac{\partial \Omega_x}{\partial x_2}
\end{vmatrix} =
\begin{vmatrix}
		\displaystyle \sqrt{\frac{2}{\mu}} & 0 \\
	0 & \displaystyle \sqrt{\frac{2}{\mu R^2}}
\end{vmatrix} = 2 \sqrt{\frac{1}{\mu}} \, \sqrt{\frac{1}{\mu R^2}} = \frac{2}{\mu R} \notag \\
\left[ \, Jac \, \right]_{ham} = \Bigg{|} \frac{\partial ( p_R, J_x )}{\partial x_1, x_2 )} \Bigg{|} = 
\begin{vmatrix}
		\displaystyle \frac{\partial p_R}{\partial x_1} & \displaystyle \frac{\partial p_R}{\partial x_2} \\[2ex]
	\displaystyle \frac{\partial J_x}{\partial x_1} & \displaystyle \frac{\partial J_x}{\partial x_2}
\end{vmatrix} =
\begin{vmatrix}
	\sqrt{2 \mu} & 0 \\
	0 & \sqrt{2 \mu R^2}
\end{vmatrix} = \sqrt{2 \mu} \, \sqrt{2 \mu R^2} = 2 \mu R \notag
\end{gather}

Отметим, что якобианы связаны следующим соотношением
\begin{gather}
	\JacL \cdot \JacH = 2^2 \notag	
\end{gather}

\section{Система CO$_2-$Ar}

Используем стандартную молекулярную систему отсчета для этой системы. Направим ось $OZ$ вдоль линии $C-Ar$, ось $OX$ перпендикулярно ей в плоскости системы. 

\begin{figure}[h]
\centering
\begin{tikzpicture}
	[oxygen/.style={ball color = red, circle, text = white}, 
	carbon/.style={ball color = black!30, circle, text = white},
	argon/.style={ball color = blue, circle, text = white}]
	\node (z1) at (4.5, 3) {OZ};
	\node (z2) at (-1.5, -1) {}
		edge [->, thick] (z1);

	\node (x1) at (-1, 3.5) {OX};
	\node (x2) at (3.5, -1) {}
		edge [->, thick] (x1); 

	\node (carbon) [carbon] {C};
	\node (oxygen1) [oxygen, below right of = carbon] {O}
		edge [double, thick] (carbon);
	\node (oxygen2) [oxygen, above left of = carbon] {O}
		edge [double, thick] (carbon);
	\node[argon] (argon) at (3, 2) {Ar};

	\begin{scope}[on background layer]
	\node [fill=yellow!30,fit=(z1) (z2) (x1) (x2) (carbon) (oxygen1) (oxygen2)] {};
\end{scope}
\end{tikzpicture}
\caption{Молекулярная система отсчета для системы $Ar-CO_2$.}
\end{figure}

Лагранжиан в описанной молекулярной системе отсчета имеет вид
\begin{gather}
		\mL = \frac{1}{2} \mu_2 \dR^2 + \frac{1}{2} \mu_1 l^2 \dT^2 + \mu_1 l^2 \Omega_y^2 \dT + \frac{1}{2} \lb \mu_1 l^2 \cos^2 \Theta + \mu_2 R^2 \rb \Omega_x^2 + \frac{1}{2} \lb \mu_1 l^2 + \mu_2 R^2 \rb \Omega_y^2 + \notag \\ 
	+ \frac{1}{2} \mu_1 l^2 \Omega_z^2 \sin^2 \Theta - \mu_1 l^2 \Omega_x \Omega_z \sin \Theta \cos \Theta, \notag
\end{gather}

где приведенные массы $\mu_1, \mu_2$ связаны с массами кислорода $m_1$, аргона $m_2$ и углерода $m_3$ следующим образом
\begin{gather}
	\mu_1 = \frac{m_1}{2}, \quad \quad \quad \mu_2 = \frac{m_2 (2 m_1 + m_3)}{2 m_1 + m_2 + m_3} \notag 
\end{gather}

Будем использовать следующую замену для диагонализации лагранжиана
\begin{gather}
	\begin{aligned}
			x_1^2 &= \frac{1}{2} \mu_2 \dR^2 \\
			x_2^2 &= \frac{1}{2} \mu_1 l^2 \lb \dT + \Omega_y \rb^2 \\
			x_3^2 &= \frac{1}{2} \mu_2 R^2 \Omega_y^2 \\
			x_4^2 &= \frac{1}{2} \mu_1 l^2 \lb \Omega_x \cos \Theta - \Omega_z \sin \Theta \rb^2 \\
			x_5^2 &= \frac{1}{2} \mu_2 R^2 \Omega_x^2
	\end{aligned} \label{co2ar1}
\end{gather}

Следующим шагом процедуры является получение выражений лагранжевых и гамильтоновых переменных через $x_1, \dots, x_5$. Для нахождения первых решим систему линейных соотношений \eqref{co2ar1} относительно $\dR, \dT, \Omega_x, \Omega_y$ и $\Omega_z$.   
\begin{gather}
	\begin{aligned}
			\dR &= x_1 \sqrt{\frac{2}{\mu_2}} \\
			\dT &= x_2 \sqrt{\frac{2}{\mu_1 l^2}} - x_3 \sqrt{\frac{2}{\mu_2 R^2}} \\
			\Omega_x &= x_5 \sqrt{\frac{2}{\mu_2 R^2}} \\
			\Omega_y &= x_3 \sqrt{\frac{2}{\mu_2 R^2}} \\
			\Omega_z &= x_5 \sqrt{\frac{2}{\mu_2 R^2}} \cot{\Theta} - \frac{x_4}{\sin \Theta} \sqrt{\frac{2}{\mu_1 l^2}} 
	\end{aligned} \notag
\end{gather}

Действуя блочной матрицей $\bbB$ на вектор лагранжевых переменных, получаем вектор гамильтоновых переменных. Подставляя в это выражение лагранжевы переменные через $x_1, \dots, x_5$, мы получаем связь гамильтоновых перменных и $x_1, \dots, x_5$:
\begin{gather}
	\begin{aligned}
	\mf{J} &= \frac{\partial T_\mL}{\partial \mf{\Omega}} = \bbI \mf{\Omega} + \bbA \dot{\mf{q}} \\
		\mf{p} &= \frac{\partial T_\mL}{\partial \dot{\mf{q}}} = \bbA^\top \mf{\Omega} + \bba \dot{\mf{q}}
	\end{aligned}
	\quad \implies \quad
	\begin{bmatrix}
		\mf{J} \\
		\mf{p}
	\end{bmatrix} =
	\begin{bmatrix}
		\bbI & \bbA \\
		\bbA^\top & \bba
	\end{bmatrix}
	\begin{bmatrix}
		\mf{\Omega} \\
		\dot{\mf{q}}
	\end{bmatrix} =
	\bbB
	\begin{bmatrix}
		\mf{\Omega} \\
		\dot{\mf{q}}
	\end{bmatrix} \\
	\begin{aligned}
		p_R &= \mu_2 \dR \\
		\pT &= \mu_1 l^ 2\lb \dT + \Omega_y \rb \\
		J_x &= \lb \mu_1 l^2 \cos^2 \Theta + \mu_2 R^2 \rb \Omega_x - \mu_1 l^2 \Omega_z \sin \Theta \cos \Theta \\
		J_y &= \mu_1 l^2 \dT + \lb \mu_1 l^2 + \mu_2 R^2 \rb \Omega_y \\
		J_z &= - \mu_1 l^2 \Omega_x \sin \Theta \cos \Theta + \mu_1 l^2 \Omega_z \sin^2 \Theta 
	\end{aligned} 
	\quad \implies \quad 
	\begin{aligned}
		p_R &= x_1 \sqrt{2 \mu_2} \\
		\pT &= x_2 \sqrt{2 \mu_1 l^2} \\
		J_x &= x_4 \sqrt{2 \mu_1 l^2} \cos \Theta + x_5 \sqrt{2 \mu_2 R^2} \\
		J_y &= x_2 \sqrt{2 \mu_1 l^2} + x_3 \sqrt{2 \mu_2 R^2} \\
		J_z &= - x_4 \sqrt{2 \mu_1 l^2} \sin \Theta
\end{aligned} \label{co2ar2}
\end{gather}

Получим выражение для кинетической энергии в гамильтоновой форме
\begin{gather}
	T_\mH = \dR p_R + \dT \pT + \Omega_x J_x + \Omega_y J_y + \Omega_z J_z - T_\mL = \left[ 2 x_1^2 \right] + \left[ 2 x_2^2 - 2 x_2 x_3 \sqrt{\frac{\mu_1 l^2}{\mu_2 R^2}} \right] + \notag \\
	+ \left[ 2 x_5^2 + 2 x_4 x_5 \cos \Theta \sqrt{\frac{\mu_1 l^2}{\mu_2 R^2}} \right] + \left[ 2 x_3^2 + 2 x_2 x_3 \sqrt{\frac{\mu_1 l^2}{\mu_2 R^2}} \right] + \left[ 2 x_4^2 - 2 x_4 x_5 \cos \Theta \sqrt{\frac{\mu_1 l^2}{\mu_2 R^2}} \right] - T_\mL = \notag \\
	= 2 \left( x_1 ^2 + \ldots + x_5^2 \right) - \left( x_1^2 + \ldots + x_5^2 \right) = x_1^2 + \ldots + x_5^2 \notag   
\end{gather}

Используя левый столбец формул в \eqref{co2ar2}, выразим лагранжевы переменные через гамильтоновы, чтобы использовать их для преобразования выражений $x_1, \dots x_5$.
\begin{gather}
	\begin{aligned}
		\dR &= \frac{1}{\mu_2} p_R \\
		\dT &= - \frac{1}{\mu_2 R^2} J_y + \pT \lb \frac{1}{\mu_2 R^2} + \frac{1}{\mu_1 l^2} \rb \\
		\Omega_x &= \frac{1}{\mu_2 R^2} J_x + \frac{\cot \Theta}{\mu_2 R^2} J_z \\
		\Omega_y &= \frac{J_y - \pT}{\mu_2 R^2} \\
		\Omega_z &= \frac{\cot \Theta}{\mu_2 R^2} J_x + \frac{\cot^2 \Theta}{\mu_2 R^2} J_z + \frac{1}{\mu_1 l^2 \sin^2 \Theta} J_z
	\end{aligned} \notag \\
	\begin{aligned}
		x_1^2 &= \frac{1}{2} \mu_2 \dR^2 \\ 
		x_2^2 &= \frac{1}{2} \mu_1 l^2 \lb \dT + \Omega_y \rb^2 \\ 
		x_3^2 &= \frac{1}{2} \mu_2 R^2 \Omega_y^2 \\
		x_4^2 &= \frac{1}{2} \mu_1 l^2 \lb \Omega_x \cos \Theta - \Omega_z \sin \Theta \rb^2 \\	
		x_5^2 &= \frac{1}{2} \mu_2 R^2 \Omega_x^2 
	\end{aligned}
	\quad \Longleftrightarrow \quad \quad \quad 
	\begin{aligned}
		x_1^2 &= \frac{1}{2 \mu_2} p_R^2 \\
		x_2^2 &= \frac{1}{2 \mu_1 l^2} \pT^2 \\ 
		x_3^2 &= \frac{1}{2 \mu_2 R^2} \lb \pT - J_y \rb^2 \\
		x_4^2 &= \frac{1}{2 \mu_1 l^2 \sin^2 \Theta} J_z^2 \\ 
		x_5^2 &= \frac{1}{2 \mu_2 R^2} \lb J_x + J_z \cot \Theta \rb^2 
	\end{aligned} \notag \\
	T_\mL = x_1^2 + x_2^2 + x_3^2 + x_4^2 + x_5^2 \quad \quad \quad \Longleftrightarrow \quad \quad \quad T_\mH = x_1^2 + x_2^2 + x_3^2 + x_4^2 + x_5^2 \notag 
\end{gather}

Выпишем выражения для якобианов замены переменных в лагранжиане и гамильтониане
\begin{gather}
	\JacL = \Bigg{|} \frac{\partial (\dR, \dT, \Omega_x, \Omega_y, \Omega_z )}{\partial (x_1, x_2, x_3, x_4, x_5)} \Bigg{|} =
	\begin{vmatrix}
			\displaystyle \sqrt{\frac{2}{\mu_2}} & 0 & 0 & 0 & 0 \\[2ex]
			0 & \displaystyle \sqrt{\frac{2}{\mu_1 l^2}} & -\displaystyle \sqrt{\frac{2}{\mu_2 R^2}} & 0 & 0 \\[2ex]
			0 & 0 & 0 & 0 & \displaystyle \sqrt{\frac{2}{\mu_2 R^2}} \\[2ex]
			0 & 0 & \displaystyle \sqrt{\frac{2}{\mu_2 R^2}} & 0 & 0 \\[2ex]
		0 & 0 & 0 & -\displaystyle \sqrt{\frac{2}{\mu_1 l^2 \sin^2 \Theta}} & \displaystyle \sqrt{\frac{2}{\mu_2 R^2}} \cot \Theta 
	\end{vmatrix} = \notag \\
	= 2^{5/2} \displaystyle \sqrt{\frac{1}{\mu_2}} \sqrt{\frac{1}{\mu_1 l^2}} \sqrt{\frac{1}{\mu_1 l^2 \sin^2 \Theta}} \frac{1}{\mu_2 R^2} = 2^{5/2} \mu_1^{-1} \mu_2^{-3/2} l^{-2} R^{-2} \sin^{-1} \Theta \notag \\
	\JacH = \Bigg{|} \frac{\partial (p_R, \pT, J_x, J_y, J_z)}{\partial (x_1, x_2, x_3, x_4, x_5)} \Bigg{|} = 
	\begin{vmatrix}
		\displaystyle \sqrt{2 \mu_2} & 0 & 0 & 0 & 0 \\
		0 & \displaystyle \sqrt{2 \mu_1 l^2} & 0 & 0 & 0 \\
		0 & 0 & 0 & \displaystyle \sqrt{2 \mu_2 R^2} & - \displaystyle \sqrt{2 \mu_1 l^2} \cos \Theta \\
		0 & \displaystyle \sqrt{2 \mu_1 l^2} & - \displaystyle \sqrt{2 \mu_2 R^2} & 0 & 0 \\
		0 & 0 & 0 & 0 & \displaystyle \sqrt{2 \mu_1 l^2} \sin \Theta 
	\end{vmatrix} = \notag \\
	= 2^{5/2} \mu_1 \mu_2^{3/2} l^2 R^2 \sin \Theta \notag
\end{gather}

Якобианы связаны следующим соотношением
\begin{gather}
	\JacL \cdot \JacH = 2^5 \notag
\end{gather}

\section{Система N$_2-$N$_2$}

%Omega[x]^2*mu[2]*l[2]^2 + 
%Omega[x]^2*mu[3]*q[4]^2 + 
%Omega[y]^2*mu[1]*l[1]^2 + 
%Omega[y]^2*mu[3]*q[4]^2 + 
%Omega[z]^2*mu[2]*l[2]^2 + 2.*Omega[x]*Omega[y]*mu[2]*l[2]^2*cos(q[2])*sin(q[2])*cos(q[3])^2 - 
%2.*v[2]*Omega[x]*mu[2]*l[2]^2*cos(q[2])*sin(q[3])*cos(q[3]) - 
%2.*v[2]*Omega[y]*mu[2]*l[2]^2*sin(q[2])*sin(q[3])*cos(q[3]) - 
%2.*Omega[x]*Omega[z]*mu[2]*l[2]^2*cos(q[2])*sin(q[3])*cos(q[3]) - 
%2.*Omega[y]*Omega[z]*mu[2]*l[2]^2*sin(q[2])*sin(q[3])*cos(q[3]) - 
%2.*Omega[x]*Omega[z]*mu[1]*l[1]^2*sin(q[1])*cos(q[1]) - 
%2.*Omega[x]*Omega[y]*mu[2]*l[2]^2*cos(q[2])*sin(q[2]) + 
%2.*v[2]*Omega[z]*mu[2]*l[2]^2 + 
%v[2]^2*mu[2]*l[2]^2 + 
%Omega[x]^2*mu[1]*l[1]^2*cos(q[1])^2 - 
%1.*Omega[x]^2*mu[2]*l[2]^2*cos(q[2])^2 + 
%Omega[y]^2*mu[2]*l[2]^2*cos(q[2])^2 + 
%Omega[y]^2*mu[2]*l[2]^2*cos(q[3])^2 - 
%1.*Omega[z]^2*mu[1]*l[1]^2*cos(q[1])^2 - 
%1.*Omega[z]^2*mu[2]*l[2]^2*cos(q[3])^2 - 
%1.*v[2]^2*mu[2]*l[2]^2*cos(q[3])^2 + 
%2.*v[1]*Omega[y]*mu[1]*l[1]^2 + 
%v[1]^2*mu[1]*l[1]^2 + 
%Omega[z]^2*mu[1]*l[1]^2 + 
%mu[3]*v[4]^2 + 
%Omega[x]^2*mu[2]*l[2]^2*cos(q[2])^2*cos(q[3])^2 - 
%1.*Omega[y]^2*mu[2]*l[2]^2*cos(q[2])^2*cos(q[3])^2 - 
%2.*v[2]*Omega[z]*mu[2]*l[2]^2*cos(q[3])^2 - 
%2.*v[3]*Omega[x]*mu[2]*l[2]^2*sin(q[2]) +
%%%%%%%%%%%%%%%%%%%%%%%%%%%%%%%%%%%%%%%%%%%%%%
% 2.*v[3]*Omega[y]*mu[2]*l[2]^2*cos(q[2]) + 
% v[3]^2*mu[2]*l[2]^2

Используем следующую замену для диагонализации лагранжиана
\begin{gather}
	\begin{aligned}
			x_1^2 &= \frac{1}{2} \mu_2 l_2^2 \lb - \dot{\varphi} \sin \theta_2 + \Omega_x \cos \varphi \cos \theta_2 + \Omega_y \sin \varphi \cos \theta_2 - \Omega_z \sin \theta_2 \rb^2 \\
			x_2^2 &= \frac{1}{2} \mu_2 l_2^2 \lb \dot{\theta}_2 - \Omega_x \sin \varphi + \Omega_y \cos \varphi \rb^2 \\
			x_3^2 &= \frac{1}{2} \mu_1 l_1^2 \lb \Omega_x \cos \theta_1 - \Omega_z \sin \theta_1 \rb^2 \\
			x_4^2 &= \frac{1}{2} \mu_3 R^2 \Omega_x^2 \\
			x_5^2 &= \frac{1}{2} \mu_3 R^2 \Omega_y^2 \\
			x_6^2 &= \frac{1}{2} \mu_1 l_1^2 \lb \dot{\theta}_1  + \Omega_y \rb^2 \\
			x_7^2 &= \frac{1}{2} \mu_3 \dot{R}^2 
	\end{aligned} \label{n2n2_1}
\end{gather}

Выражаем лагранжевы переменные через $x_1, \dots, x_7$:
\begin{gather}
	\begin{aligned}
			\dtone &= x_6 \sqrt{\frac{2}{\mu_1 l_1^2}} - x_5 \sqrt{\frac{2}{\mu_3 R^4}} \\ 
		\dphi &= - \frac{x_1}{\sin \Theta_2} \sqrt{\frac{2}{\mu_2 l_2^2}} + \frac{x_4 \cos \varphi \cos \Theta_2}{\sin \Theta_2} \sqrt{\frac{2}{\mu_3 R^2}} + \frac{x_5 \sin \varphi \cos \Theta_2}{\sin \Theta_2} \sqrt{\frac{2}{\mu_3 R^2}} + \frac{x_3}{\sin \Theta_1} \sqrt{\frac{2}{\mu_1 l_1^2}}
	- \frac{x_4 \cos \Theta_1}{\sin \Theta_1} \sqrt{\frac{2}{\mu_3 R^2}} \\
	\dttwo &= x_2 \sqrt{\frac{2}{\mu_2 l_2^2}} + x_4 \sin \varphi \sqrt{\frac{2}{\mu_3 R^2}} - x_5 \cos \varphi \sqrt{\frac{2}{\mu_3 R^2}} \\
	\dR &= x_7 \sqrt{\frac{2}{\mu_3}} \\
	\Omega_x &= x_4 \sqrt{\frac{2}{\mu_3 R^2}} \\
	\Omega_y &= x_5 \sqrt{\frac{2}{\mu_3 R^2}} \\
	\Omega_z &= - \frac{x_3}{\sin \Theta_1} \sqrt{\frac{2}{\mu_1 l_1^2}} + \frac{x_4 \cos \Theta_1}{\sin \Theta_1} \sqrt{\frac{2}{\mu_3 R^2}}
	\end{aligned} \notag
\end{gather}

Действуем матрицей $\bbB$ на вектор лагранжевых переменных и подставляем их выражения через $x_1, \dots x_7$:
\begin{gather}
	\begin{aligned}
		J_x &= x_4 \sqrt{2 \mu_3 R^2} + x_3 \sqrt{2 \mu_1 l_1^2} \cos \Theta_1 + x_1 \sqrt{2 \mu_2 l_2^2} \cos \varphi \cos \Theta_2 - x_2 \sqrt{2 \mu_2 l_2^2} \sin \varphi \\
		J_y &= x_5 \sqrt{2 \mu_3 R^2} + x_6 \sqrt{2 \mu_1 l_1^2} + x_1 \sqrt{2 \mu_2 l_2^2} \sin \varphi \cos \Theta_2 + x_2 \sqrt{2 \mu_2 l_2^2} \cos \varphi\\
		J_z &= - x_3 \sqrt{2 \mu_1 l_1^2} \sin \Theta_1 - x_1 \sqrt{2 \mu_2 l_2^2} \sin \Theta_2 \\
		\ptone &= x_6 \sqrt{2 \mu_1 l_1^2} \\
		\pphi &= - x_1 \sqrt{2 \mu_2 l_2^2} \sin \Theta_2 \\
		\pttwo &= x_2 \sqrt{2 \mu_2 l_2^2} \\
		p_R &= \sqrt{2 \mu_3} x_7
	\end{aligned} \notag
\end{gather}

Правильность полученных выражений подтверждается следующей связью
\begin{gather}
		T_\mH = \Omega_x J_x + \Omega_y J_y + \Omega_z J_z + \ptone \dtone + \pttwo \dttwo + \pphi \dot{\varphi} + p_R \dot{R} - T_\mL = \notag \\
	= \left[ 2 x_1 ^2 + \ldots + 2 x_7^2 \right] - \left[ x_1^2 + \ldots + x_7^2 \right] = x_1^2 + \ldots + x_7^2 \notag
\end{gather}

Выражения, свящывающие лагранжевы переменные с гамильтоновыми приводить здесь не буду, они слишком громоздки, однако руками мне удалось получить компактный вид для 6 из 7 выражений. Подставив эти выражения в \eqref{n2n2_1}, получаем новые переменные $x_1, \ldots, x_7$ в гамильтоновых переменных:
\begin{gather}
	\begin{aligned}
		x_1 &= - \frac{1}{\sqrt{2 \mu_2 l_2^2} \sin \Theta_2} p_\varphi \\
		x_2 &= \frac{1}{\sqrt{2 \mu_2 l_2^2}} p_{\Theta_2} + \frac{\sqrt{2 \mu_2 l_2^2}}{\mu_3 R^2} p_{\Theta_1} \cot^2 \Theta_1 \sin \varphi \cos \Theta_2 \\
		x_3 &= \frac{1}{\sqrt{2 \mu_1}} \frac{J_z - p_\varphi}{\sin \Theta_1} \\
		x_4 &= \frac{1}{\sqrt{2 \mu_3 R^2}} \lb p_{\Theta_2} \sin \varphi + J_x + J_z \cos \Theta_1 + p_\varphi \cot \Theta_2 \cos \varphi - p_\varphi \cot \Theta_1 \rb \\
		x_5 &= \frac{1}{\sqrt{2 \mu_3 R^2}} \lb p_\varphi \sin \varphi \cot \Theta_2 + J_y - p_{\Theta_2} \cos \varphi - p_{\Theta_1} \rb \\
		x_6 &= \frac{1}{\sqrt{2 \mu_1 l_1^2}} p_{\Theta_1} \\
		x_7 &= \frac{1}{\sqrt{2 \mu_3}} p_R
	\end{aligned}
\end{gather}

Несмотря на значительный размер выражений, якобианы замены переменных как вслучае лагранжиана так и в случае гамильтониана могут быть получены в компактном виде без дополнительных упрощений руками. Представлю конечный результат:
\begin{gather}
		\JacL = \Bigg{|} \frac{\partial ( \dR, \dot{\Theta}_1, \dot{\Theta}_2, \dot{\varphi}, \Omega_x, \Omega_y, \Omega_z )}{\partial ( x_1, x_2, x_3, x_4, x_5, x_6, x_7 0} \Bigg{|} = 2^{7/2} \mu_1^{-1} l_1^{-2} \mu_2^{-1} l_2^{-2} \mu_3^{-3/2} R^{-2} \sin^{-1} \Theta_1 \sin^{-1} \Theta_2 \notag \\
		\JacH = \Bigg{|} \frac{\partial ( p_R, p_{\Theta_1}, p_{\Theta_2}, p_\varphi, J_x, J_y, J_z )}{\partial ( x_1, x_2, x_3, x_4, x_5, x_6, x_7) } \Bigg{|} = 2^{7/2} \mu_1 l_1^2 \mu_2 l_2^2 \mu_3^{3/2} R^2 \sin \Theta_1 \sin \Theta_2 \notag \\
		\JacL \cdot \JacH = 2^7 \notag
\end{gather}

\end{document}
