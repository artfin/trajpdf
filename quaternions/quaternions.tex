\documentclass[14pt]{extarticle}

\usepackage[T1]{fontenc}
\usepackage[utf8]{inputenc}
\usepackage[russian]{babel}

% page margin
\usepackage[top=2cm, bottom=2cm, left=0.5cm, right=0.5cm]{geometry}

% AMS packages
\usepackage{amsmath, array}
\usepackage{amssymb}
\usepackage{amsfonts}
\usepackage{amsthm}

\usepackage{mathtools}

\usepackage{graphicx}

\usepackage{fancyhdr}
\pagestyle{fancy}
% modifying page layout using fancyhdr
\fancyhf{}
\renewcommand{\sectionmark}[1]{\markright{\thesection\ #1}}
\renewcommand{\subsectionmark}[1]{\markright{\thesubsection\ #1}}

\rhead{\fancyplain{}{\rightmark }}
\cfoot{\fancyplain{}{\thepage }}

\usepackage{titlesec}
\titleformat{\section}{\bfseries}{\thesection.}{1em}{}
\titleformat{\subsection}{\normalfont\itshape\bfseries}{\thesubsection.}{0.5em}{}


\makeatletter
\def\env@dmatrix{\hskip -\arraycolsep
  \let\@ifnextchar\new@ifnextchar
  \extrarowheight=2ex
  \array{*\c@MaxMatrixCols{>{\displaystyle}c}}}

\newenvironment{dmatrix}
  {\env@dmatrix}
  {\endarray\hskip-\arraycolsep}

\newenvironment{bdmatrix}
  {\left[\env@dmatrix}
  {\endmatrix\right]}
% and other matrix environments are similar
\makeatother

\newcommand{\bbS}{\mathbb{S}}

\newcommand{\ao}{\alpha_1}
\newcommand{\bo}{\beta_1}
\newcommand{\go}{\gamma_1}
\newcommand{\at}{\alpha_2}
\newcommand{\bt}{\beta_2}
\newcommand{\gt}{\gamma_2}
\newcommand{\ath}{\alpha_3}
\newcommand{\bth}{\beta_3}
\newcommand{\gth}{\gamma_3}

\newcommand{\lb}{\left(}
\newcommand{\rb}{\right)}
\newcommand{\lsq}{\left[}
\newcommand{\rsq}{\right]}

\begin{document}
\section{Определение углов Эйлера вращения, являющегося композицией двух вращений}

Рассмотрим композицию двух вращений $\bbS_1$ и $\bbS_2$, параметризованновых наборами углов Эйлера $(\ao, \bo, \go)$ и $(\at, \bt, \gt)$, соответственно, определенными в Голдстейне (zxz; 313 extrinsic). 
\begin{gather}
	\bbS_1(\ao, \bo, \go) \cdot \bbS_2(\at, \bt, \gt) = \bbS_3(\ath, \bth, \gth) 
\end{gather}

Перейдем от представления вращений при помощи эйлеровых углов к кватернионному, которое позволяет более удобным образом описывать параметры вращения, являющегося композицией вращений. Рассчитав компоненты кватерниона, соотвествующего композиции вращений, через наборы углов Эйлера 1 и 2, выразим через них углы Эйлера результирующего вращения. Компоненты кватерниона $q_1$, соответствующего вращению 1, связаны с углами Эйлера следующими соотношениями:
\begin{gather}
q_1 = 
\begin{bdmatrix}
	\cos \frac{\ao - \go}{2} \sin \frac{\bo}{2} \\
	\sin \frac{\ao - \go}{2} \sin \frac{\bo}{2} \\
	\sin \frac{\ao + \go}{2} \cos \frac{\bo}{2} \\
	\cos \frac{\ao + \go}{2} \cos \frac{\bo}{2}
\end{bdmatrix} \label{def}
\end{gather} 

Кватернионы представляют в виде пары $[$действительное число, вектор$]$: $q_1 = \lsq q_1^0, \mathbf{q}_1 \rsq$. Произведение кватернионов в векторной форме представлено соотношением
\begin{gather}
	q_1 \cdot q_2 = \lb q_1^0 q_2^0 - \mathbf{q}_1 \mathbf{q_2} \rb + q_1^0 \mathbf{q}_2 + q_2^0 \mathbf{q}_1 + \lsq \mathbf{q}_1 \times \mathbf{q}_2 \rsq \label{prod}
\end{gather}

Подставив выражения для $q_1$ и $q_2$ в определение $\eqref{prod}$, получаем выражение для компонент кватерниона $q_3 = q_1 \cdot q_2$:
\begin{gather}
	q_3 =
	\begin{bdmatrix}
		\sin \frac{\bo}{2} \sin \frac{\bt}{2} \cos \lb \frac{\ao - \go}{2} + \frac{\at - \gt}{2} \rb - \cos \frac{\bo}{2} \cos \frac{\bt}{2} \cos \lb \frac{\ao + \go}{2} - \frac{\at + \gt}{2} \rb \\
		\sin \frac{\bo}{2} \sin \frac{\bt}{2} \sin \lb \frac{\ao - \go}{2} + \frac{\at - \gt}{2} \rb + \cos \frac{\bo}{2} \cos \frac{\bt}{2} \sin \lb \frac{\ao + \go}{2} - \frac{\at + \gt}{2} \rb \\
		\sin \frac{\bo}{2} \cos \frac{\bt}{2} \sin \lb \frac{\at + \gt}{2} - \frac{\ao - \go}{2} \rb + \cos \frac{\bo}{2} \sin \frac{\bt}{2} \sin \lb \frac{\ao + \go}{2} + \frac{\at - \gt}{2} \rb \\
		\sin \frac{\bo}{2} \cos \frac{\bt}{2} \cos \lb \frac{\ao - \go}{2} - \frac{\at + \gt}{2} \rb + \cos \frac{\bo}{2} \sin \frac{\bt}{2} \cos \lb \frac{\ao + \go}{2} + \frac{\at - \gt}{2} \rb
	\end{bdmatrix}
\end{gather}

Углы Эйлера $\lb \ath, \bth, \gth \rb$ связаны с компонентами кватерниона $q_3$ соотношениями
\begin{gather}
	\tan \ath = \frac{ (q_3)_1 (q_3)_3 + (q_3)_0 (q_3)_2}{ (q_3)_0 (q_3)_1 - (q_3)_2 (q_3)_3} \\
	\cos \bth = (q_3)_0^2 + (q_3)_3^2 - (q_3)_1^2 - (q_3)_2^2 \\
	\tan \gth = \frac{ (q_3)_1 (q_3)_3 - (q_3)_0 (q_3)_2}{ (q_3)_2 (q_3)_3 + (q_3)_0 (q_3)_1}
\end{gather} 

Опуская промежуточные выкладки, приходим к следующим выражениям
\begin{gather}
	u = (\cos \bt \sin \bo + \cos \bo \sin \bt \cos ( \gt - \ao) ) \sin \go - \sin \bt \cos \go \sin(\gt - \ao) \notag \\
	v = \sin (\gt - \ao) \Big[ \sin \at \cos \bo \sin \go + \cos \at \cos \bt \cos \go \Big] + \notag \\
	+ \cos (\gt - \ao) \Big[ \sin \at \cos \go - \cos \bo \cos \bt \cos \at \sin \go \Big] + \sin \go \cos \at \sin \bo \sin \bt \notag \\
	\tan \gth = \frac{u}{v} \notag \\
	s = ( \sin \bo \cos \bt \cos ( \gt - \ao ) + \cos \bo \sin \bt ) \sin \at + \sin \bo \cos \at \sin ( \gt - \ao ) \notag \\
	t = - \sin ( \gt - \ao) \Big[ \cos \bt \sin \go \sin \at + \cos \bo \cos \go \cos \at \Big] + \notag \\
	+ \cos ( \gt - \ao ) \Big[ \sin \go \cos \at - \cos \bo \cos \bt \cos \go \sin \at \Big] +  \sin \bo \sin \bt \cos \go \sin \at \notag \\
	\tan \ath = \frac{s}{t} \notag \\
	\cos \bth = - \sin \go \sin \at \sin \bo \sin \bt + (\sin \go \sin \at \cos \bo \cos \bt + \cos \go \cos \at) \cos (\ao - \gt) + \notag \\
	+ (\sin \at \cos \bt \cos \go - \sin \go \cos \at \cos \bo) \sin (\ao - \gt) \notag
\end{gather}

\end{document}