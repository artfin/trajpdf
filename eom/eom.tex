\documentclass[12pt]{article}

\usepackage[T1]{fontenc}
\usepackage[utf8]{inputenc}
\usepackage[russian]{babel}

% page margin
\usepackage[top=2cm, bottom=2cm, left=2cm, right=2cm]{geometry}

% AMS packages
\usepackage{amsmath}
\usepackage{amssymb}
\usepackage{amsfonts}
\usepackage{amsthm}

\usepackage{bbm}

\usepackage{array}
\usepackage{graphicx}

\usepackage{fancyhdr}
\pagestyle{fancy}
% modifying page layout using fancyhdr
\fancyhf{}
\renewcommand{\sectionmark}[1]{\markright{\thesection\ #1}}
\renewcommand{\subsectionmark}[1]{\markright{\thesubsection\ #1}}

\rhead{\fancyplain{}{\rightmark }}
\cfoot{\fancyplain{}{\thepage }}

\usepackage{titlesec}
\titleformat{\section}{\bfseries}{\thesection.}{1em}{}
\titleformat{\subsection}{\normalfont\itshape\bfseries}{\thesubsection.}{0.5em}{}

\newcommand{\mf}{\mathbf}

\newcommand{\lb}{\left(}
\newcommand{\rb}{\right)}

\newcommand{\bbI}{\mathbb{I}}
\newcommand{\bba}{\mathbbm{a}}
\newcommand{\bbA}{\mathbb{A}}
\newcommand{\bbG}{\mathbb{G}}
\newcommand{\bbW}{\mathbb{W}}
\newcommand{\EOmega}{\boldsymbol{\Omega}_{\rm \textbf{e}}}
\newcommand{\pe}{\mf{p_e}}

\def\hh{_{\rm H}}

\makeatletter
\def\env@dmatrix{\hskip -\arraycolsep
  \let\@ifnextchar\new@ifnextchar
  \extrarowheight=2ex
  \array{*\c@MaxMatrixCols{>{\displaystyle}c}}}

\newenvironment{dmatrix}
  {\env@dmatrix}
  {\endarray\hskip-\arraycolsep}

\newenvironment{bdmatrix}
  {\left[\env@dmatrix}
  {\endmatrix\right]}
% and other matrix environments are similar
\makeatother

\begin{document}

Запишем гамильтониан в следующей форме
\begin{gather}
	H = \frac{1}{2} \mf{p}^\top \bbG_{22} \lb \mf{q} \rb \mf{p} + \mf{p_e^\top} \bbW(\EOmega)^\top \, \bbG_{12} \lb \mf{q} \rb \mf{p} + \frac{1}{2} \mf{p_e^\top} \bbW(\EOmega)^\top \bbG_{11} \lb \mf{q} \rb \, \bbW(\EOmega) \mf{p_e} + U(\mf{q}), \label{eq:hamiltonian}
\end{gather}
где $\mf{q}, \mf{p}$ -- векторы внутренних координат и сопряженных им импульсов, $\EOmega, \mf{p_e}$ -- векторы эйлеровых углов и сопряженных к ним импульсов. Матрица $\bbW(\EOmega)$ связывает векторы углового момента $\mf{J}$ и эйлеровых импульсов $\mf{p_e}$ соотношением
\begin{gather}
	\mf{J} = \bbW \mf{p_e}, \quad \bbW = 
	\begin{bdmatrix}
		\frac{\sin \psi}{\sin \theta} & \cos \psi & - \frac{\cos \theta \sin \psi}{\sin \theta} \\
		\frac{\cos \psi}{\sin \theta} & - \sin \psi & - \frac{\cos \theta \cos \psi}{\sin \theta} \\
		0 & 0 & 1
	\end{bdmatrix}. 
\end{gather}

Матрицы $\bbG_{11}$, $\bbG_{12}$, $\bbG_{22}$ связаны с матрицами, опредялющими кинетическую энергию в Лагранжевой форме, следующими соотношениями:
\begin{gather}
	\begin{aligned}
		\bbG_{11} &= \lb \bbI - \bbA \bba^{-1} \bbA^\top \rb^{-1} \\ 
		\bbG_{12} &= -\bbI^{-1} \bbA \bbG_{22} = - \bbG_{11} \bbA \bba^{-1}, \\
		\bbG_{22} &= \lb \bba - \bbA^\top \bbI^{-1} \bbA \rb^{-1}
\end{aligned}
	\label{eq:Gmatrices}
\end{gather}
где 
\begin{itemize}
	\item $\bba = \left\{ \bba_{jk} \right\}$ -- матрица относительной кинетической энергии, $\displaystyle \bba_{jk} = \sum_{i = 1}^n m_i \frac{\partial \mathbf{R}_i}{\partial q_j} \frac{\partial \mathbf{R}_i}{\partial q_k}$; 
	\item $\bbA = \left\{ \bbA_{jk} \right\}$ -- кориолисова матрица, $\displaystyle \bbA_{jk} = \sum_{i = 1}^n m_i \left[ \mathbf{R}_i \times \frac{\partial \mathbf{R}_i}{\partial q_k} \right]_j$;
	\item $\bbI = \left\{ \bbI_{jk} \right\}$ -- матрица тензора инерции.
\end{itemize}

Система динамических уравнений Гамильтона состоит из 4х векторных уравнений
\begin{gather}
	\begin{aligned}
		\mf{\dot{q}} &= \frac{\partial H}{\partial \mf{p}} \\
		\mf{\dot{p}} &= -\frac{\partial H}{\partial \mf{q}} \\
		\mf{\dot{\boldsymbol{\Omega}}}_\textbf{e} &= \frac{\partial H}{\partial \mf{p_e}} \\
		\mf{\dot{p}}_\textbf{e} &= - \frac{\partial H}{\partial \EOmega}
	\end{aligned}
\end{gather}

Рассмотрим сначала производные по эйлеровым углам $\EOmega$ и сопряженным к ним импульсам $\mf{p_e}$:
\begin{gather}
		\frac{\partial H}{\partial \EOmega} = \frac{1}{2} \mf{p_e^\top} \frac{\partial \bbW^\top}{\partial \EOmega} \bbG_{11} \bbW \pe + \mf{p_e^\top} \frac{\partial \bbW^\top}{\partial \EOmega} \bbG_{12} \mf{p} + \frac{1}{2} \mf{p_e^\top} \bbW^\top \bbG_{11} \frac{\partial \bbW}{\partial \EOmega} \pe \\
		\frac{\partial H}{\partial \pe} = \bbW^\top \bbG_{11} \bbW \pe + \bbW^\top \bbG_{12} \mf{p}
\end{gather}

Производные матрицы $\bbW$ по эйлеровым углам $\EOmega$, появляющиеся в предыдущем выражении, имеют вид
\begin{gather}
	\frac{\partial \bbW}{\partial \varphi} = 0, \quad 
	\frac{\partial \bbW}{\partial \theta} = 
	\begin{bdmatrix}
		-\frac{\cos \theta \sin \psi}{\sin^2 \theta} & 0 & \frac{\sin \psi}{\sin^2 \theta} \\
		- \frac{\cos \theta \cos \psi}{\sin^2 \theta} & 0 & \frac{\cos \psi}{\sin^2 \theta} \\
		0 & 0 & 0
	\end{bdmatrix}, \quad 
	\frac{\partial \bbW}{\partial \psi} = 
	\begin{bdmatrix}
		\frac{\cos \psi}{\sin \theta} & - \sin \psi & -\frac{\cos \theta \cos \psi}{\sin \theta} \\
		- \frac{\sin \psi}{\sin \theta} & - \cos \psi & \frac{\cos \theta \sin \psi}{\sin \theta} \\
		0 & 0 & 0
	\end{bdmatrix}
\end{gather}

Теперь перейдем к рассмотрению производных гамильтониана по векторам внутренних координат $\mf{q}$ и сопряженных к ним импульсов $\mf{p}$.
\begin{gather}
	\frac{\partial H}{\partial \mf{p}} = \bbG_{22}(\mf{q}) \mf{p} + \bbG_{12}^\top (\mf{q}) \bbW(\EOmega) \pe
\end{gather}
При дифференцировании гамильтониана по $\mf{q}$ основную сложность представляют матрицы $\bbG_{ij}(\mf{q})$. При выводе этих производных были привлечены формулы из теории функциональных матриц.
Рассмотрим дифференцируемую, обратимую матрицу $\bbA(\mf{q})$ векторного аргумента $\mf{q}$. Производная обратной матрицы $\bbA^{-1}(\mf{q})$ по переменной $\mf{q}$ связана с производной матрицы $\bbA(\mf{q})$ следующим выражением 
\begin{gather}
		\frac{\partial}{\partial \mf{q}} \bbA^{-1}(\mf{q}) = -\bbA^{-1}(\mf{q})\left[ \frac{\partial}{\partial \mf{q}} \bbA(\mf{q}) \right] \bbA^{-1} (\mf{q}). 
\end{gather}

Рассмотрим производную матрицы $\bbG_{11}(\mf{q})$ по вектору внутренних координат $\mf{q}$. 
\begin{gather}
	\frac{\partial}{\partial \mf{q}} \bbG_{11} = \frac{\partial}{\partial \mf{q}} \left[ \lb \bbI - \bbA \bba^{-1} \bbA^\top \rb^{-1} \right] = - \lb \bbI - \bbA \bba^{-1} \bbA^\top \rb^{-1} \left[ \frac{\partial}{\partial \mf{q}} \lb \bbI - \bbA \bba^{-1} \bbA^\top \rb \right] \lb \bbI - \bbA \bba^{-1} \bbA^\top \rb^{-1}, 
\end{gather}
или, более кратко,
\begin{gather}
	\frac{\partial}{\partial \mf{q}} \bbG_{11} = - \bbG_{11} \left[ \frac{\partial}{\partial \mf{q}} \lb \bbI - \bbA \bba^{-1} \bbA^\top \rb \right] \bbG_{11}. \label{eq:diffG11}
\end{gather}

При дифференцировании произведения матриц также действует правило Лейбница, используя которое раскрываем производную в (\ref{eq:diffG11}).
\begin{gather}
	\frac{\partial}{\partial \mf{q}} \bbG_{11} = -\bbG_{11} \left[ \frac{\partial \, \bbI}{\partial \mf{q}} - \frac{\partial \bbA}{\partial \mf{q}} \bba^{-1} \bbA^\top + \bbA \bba^{-1} \frac{\partial \bba}{\partial \mf{q}} \bba^{-1} \bbA^\top - \bbA \bba^{-1} \frac{\partial \bbA^\top}{\partial \mf{q}} \right] \bbG_{11}. 
\end{gather}

Аналогично, получаем выражения для производных матриц $\bbG_{12}(\mf{q})$ и $\bbG_{22}(\mf{q})$ по $\mf{q}$
\begin{gather}
	\frac{\partial}{\partial \mf{q}} \bbG_{22} = -\bbG_{22} \left[ \frac{\partial \bba}{\partial \mf{q}} - \frac{\partial \bbA^\top}{\partial \mf{q}} \bbI^{-1} \bbA + \bbA^\top \bbI^{-1} \frac{\partial \, \bbI}{\partial \mf{q}} \bbI^{-1} \bbA - \bbA^\top \bbI^{-1} \frac{\partial \bbA}{\partial \mf{q}} \right] \bbG_{22}, \\
	\frac{\partial}{\partial \mf{q}} \bbG_{12} = - \left [ \frac{\partial}{\partial \mf{q}} \bbG_{11} \right] \bbA \bba^{-1} - \bbG_{11} \frac{\partial \bbA}{\partial \mf{q}} \bba^{-1} + \bbG_{11} \bbA \bba^{-1} \frac{\partial \bba}{\partial \mf{q}} \bba^{-1} = \\
	= \bbG_{22} \left[ \frac{\partial \bba}{\partial \mf{q}} - \frac{\partial \bbA^\top}{\partial \mf{q}} \bbI^{-1} \bbA + \bbA^\top \bbI^{-1} \frac{\partial \, \bbI}{\partial \mf{q}} \bbI^{-1} \bbA - \bbA^\top \bbI^{-1} \frac{\partial \bbA}{\partial \mf{q}} \right] \bbG_{22} \bbA \bba^{-1} - \bbG_{11} \frac{\partial \bbA}{\partial \mf{q}} \bba^{-1} + \bbG_{11} \bbA \bba^{-1} \frac{\partial \bba}{\partial \mf{q}} \bba^{-1}. 
\end{gather}

Итак, получив производные матриц $\bbG_{ij}$ по $\mf{q}$, мы можем получить производные гамильтониана по $\mf{q}$ по выражению
\begin{gather}
	\frac{\partial H}{\partial \mf{q}} = \frac{1}{2} \mf{p}^\top \frac{\partial \bbG_{22}}{\partial \mf{q}} \mf{p} + \pe^\top \bbW^\top \frac{\partial \bbG_{12}}{\partial \mf{q}} \mf{p} + \frac{1}{2} \pe^\top \bbW^\top \frac{\partial \bbG_{11}}{\partial \mf{q}} \bbW \pe + \frac{\partial U}{\partial \mf{q}}.
\end{gather}

\end{document}

