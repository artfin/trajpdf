\documentclass[12pt]{article}

\usepackage[T1]{fontenc}
\usepackage[utf8]{inputenc}
\usepackage[russian]{babel}

% page margin
\usepackage[top=2cm, bottom=2cm, left=2cm, right=2cm]{geometry}

% AMS packages
\usepackage{amsmath}
\usepackage{amssymb}
\usepackage{amsfonts}
\usepackage{amsthm}

\usepackage{bbm}

\usepackage{array}
\usepackage{graphicx}

\usepackage{fancyhdr}
\pagestyle{fancy}
% modifying page layout using fancyhdr
\fancyhf{}
\renewcommand{\sectionmark}[1]{\markright{\thesection\ #1}}
\renewcommand{\subsectionmark}[1]{\markright{\thesubsection\ #1}}

\rhead{\fancyplain{}{\rightmark }}
\cfoot{\fancyplain{}{\thepage }}

\usepackage{titlesec}
\titleformat{\section}{\bfseries}{\thesection.}{1em}{}
\titleformat{\subsection}{\normalfont\itshape\bfseries}{\thesubsection.}{0.5em}{}

\newcommand{\mf}{\mathbf}

\newcommand{\lb}{\left(}
\newcommand{\rb}{\right)}

\newcommand{\bbI}{\mathbb{I}}
\newcommand{\bba}{\mathbbm{a}}
\newcommand{\bbA}{\mathbb{A}}
\newcommand{\bbG}{\mathbb{G}}

\newcommand{\Jx}{J_X}
\newcommand{\Jy}{J_Y}
\newcommand{\Jz}{J_Z}

\begin{document}

Введем следующие обозначения:
\begin{gather}
	\mu_1 = \frac{m_1}{2}, \quad 
	\mu_2 = \frac{m_2 \lb 2 m_1 + m_3 \rb}{2 m_1 + m_2 + m_3} \notag
\end{gather}

Матрицы $\bba$, $\bbA$, $\bbI$, определяющие вид кинетической энергии в лагранжевой форме: 
\begin{gather}
	\bba =
	\begin{bmatrix}
		\mu_2 & 0 \\
		0 & \mu_1 l^2
	\end{bmatrix} \quad 
	\bbA = 
	\begin{bmatrix}
		0 & 0 \\
		0 & \mu_1 l^2 \\
		0 & 0 
	\end{bmatrix} \quad
	\bbI = 
	\begin{bmatrix}
		\mu_1 l^2 \cos^2 \theta + \mu_2 R^2 & 0 & -\mu_1 l^2 \sin \theta \cos \theta \\
		0 & \mu_1 l^2 + \mu_2 R^2 & 0 \\
		- \mu_1 l^2 \sin \theta \cos \theta & 0 & \mu_1 l^2 \sin^2 \theta
	\end{bmatrix} \notag
\end{gather}

Используя формулы Фробениуса, получаем матрицы $\bbG_{11}$, $\bbG_{12}$, $\bbG_{22}$, определяющие кинетическую энергию в гамильтоновой форме:
\begin{gather}
	\bbG_{11} =
	\begin{bmatrix}
		\dfrac{1}{\mu_2 R^2} & 0 & \dfrac{1}{\mu_2 R^2 \tan \theta} \\
		0 & \dfrac{1}{\mu_2 R^2} & 0 \\
		\dfrac{1}{\mu_2 R^2 \tan \theta} & 0 & \dfrac{1}{\mu_2 R^2 \tan^2 \theta} + \dfrac{1}{\mu_1 l^2 \sin^2 \theta}
	\end{bmatrix} \quad
	\bbG_{12} =
	\begin{bmatrix}
		0 & 0 \\
		0 & - \dfrac{1}{\mu_2 R^2} \\
		0 & 0
	\end{bmatrix} \quad 
	\bbG_{22} = 
	\begin{bmatrix}
		\dfrac{1}{\mu_2} & 0 \\
		0 & \dfrac{1}{\mu_2 R^2} + \dfrac{1}{\mu_1 l^2}
	\end{bmatrix} \notag
\end{gather}

Получаем гамильтониан системы CO$_2-$Ar в заданной молекулярной системе координат:
\begin{gather}
    H = \frac{1}{2 \mu_2} p_R^2 + \lb \frac{1}{2 \mu_2 R^2} + \frac{1}{2 \mu_1 l^2} \rb p_\theta^2 - \frac{1}{\mu_2 R^2} p_\theta \Jy + \frac{1}{2 \mu_2 R^2} \Jy^2 + \frac{1}{2 \mu_2 R^2} \Jx^2 + \frac{1}{2 \sin^2 \theta} \lb \frac{\cos^2 \theta}{\mu_2 R^2} + \frac{1}{\mu_1 l^2} \rb \Jz^2 + \notag \\
+ \frac{1}{\mu_2 R^2 \tan \theta} \Jx \Jz + U(R, \theta) \notag
\end{gather}




\end{document}

