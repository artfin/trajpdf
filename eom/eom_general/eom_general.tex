\documentclass[12pt]{article}

\usepackage[T1]{fontenc}
\usepackage[utf8]{inputenc}
\usepackage[russian]{babel}

% page margin
\usepackage[top=2cm, bottom=2cm, left=2cm, right=2cm]{geometry}

% AMS packages
\usepackage{amsmath}
\usepackage{amssymb}
\usepackage{amsfonts}
\usepackage{amsthm}

\usepackage{bbm}

\usepackage{fancyhdr}
\pagestyle{fancy}
% modifying page layout using fancyhdr
\fancyhf{}
\renewcommand{\sectionmark}[1]{\markright{\thesection\ #1}}
\renewcommand{\subsectionmark}[1]{\markright{\thesubsection\ #1}}

\rhead{\fancyplain{}{\rightmark }}
\cfoot{\fancyplain{}{\thepage }}

\newcommand{\mf}{\mathbf}

\newcommand{\lb}{\left(}
\newcommand{\rb}{\right)}
\newcommand{\lsq}{\left[}
\newcommand{\rsq}{\right]}

\newcommand{\bbI}{\mathbb{I}}
\newcommand{\bba}{\mathbbm{a}}
\newcommand{\bbA}{\mathbb{A}}
\newcommand{\bbB}{\mathbb{B}}
\newcommand{\bbG}{\mathbb{G}}
\newcommand{\bbM}{\mathbb{M}}
\newcommand{\bbV}{\mathbb{V}}
\newcommand{\bbW}{\mathbb{W}}
\newcommand{\EOmega}{\boldsymbol{\Omega}_{\rm \textbf{e}}}
\newcommand{\mfpe}{\mf{p_e}}
\newcommand{\mfpet}{\mf{p_e^+}}
\newcommand{\mfq}{\mf{q}}
\newcommand{\mfp}{\mf{p}}
\newcommand{\mfJ}{\mf{J}}

\DeclareMathAlphabet{\mymathbb}{U}{BOONDOX-ds}{m}{n}
\newcommand{\bbzero}{\mymathbb{0}}
\newcommand{\bbone}{\mymathbb{1}}

\def\hh{_{\rm H}}

\usepackage{array}
\makeatletter
\def\env@dmatrix{\hskip -\arraycolsep
  \let\@ifnextchar\new@ifnextchar
  \extrarowheight=2ex
  \array{*\c@MaxMatrixCols{>{\displaystyle}c}}}

\newenvironment{dmatrix}
  {\env@dmatrix}
  {\endarray\hskip-\arraycolsep}

\newenvironment{bdmatrix}
  {\left[\env@dmatrix}
  {\endmatrix\right]}
\makeatother

\begin{document}

Рассмотрим колебательно-вращательный гамильтониан в следующей форме
\begin{gather}
    H = \frac{1}{2} \mfp^+ \bbG_{22} \mfp + \mfJ^+ \bbG_{12} \mfp + \frac{1}{2} \mfJ^+ \bbG_{11} \mfJ \notag
\end{gather}
в блочном виде
\begin{gather}
    H = \frac{1}{2} \begin{bmatrix} \mfJ^+ & \mfp^+ \end{bmatrix}
    \begin{bmatrix}
        \bbG_{11} & \bbG_{12} \\
        \bbG_{12}^+ & \bbG_{22}
    \end{bmatrix}
    \begin{bmatrix}
        \mfJ \\
        \mfp
   \end{bmatrix} = 
   \frac{1}{2} \begin{bmatrix} \mfJ^+ & \mfp^+ \end{bmatrix}
   \bbB^{-1}
   \begin{bmatrix}
        \mfJ \\
        \mfp
   \end{bmatrix}, \notag
\end{gather}
где 
\begin{gather}
    \bbB = 
    \begin{bmatrix}
        \bbI & \bbA \\
        \bbA^+ & \bba
    \end{bmatrix}. \notag
\end{gather}

В результате дифференцирования гамильтониана по блочному вектору $\displaystyle \begin{bmatrix} \mfJ \\ \mfp \end{bmatrix}$ получим блочный вектор производных $\displaystyle \begin{bdmatrix} \frac{\partial H}{\partial \mfJ} \\ \frac{\partial H}{\partial \mfp} \end{bdmatrix}$:
\begin{gather}
    \begin{bdmatrix}
        \frac{\partial H}{\partial \mfJ} \\
        \frac{\partial H}{\partial \mfp}
    \end{bdmatrix} = 
    \bbB^{-1}
    \begin{bmatrix}
        \mfJ \\
        \mfp
    \end{bmatrix}. \notag
\end{gather}

Для ясности перепишем это соотношение покомпонентно
\begin{gather}
    \begin{aligned}
        \frac{\partial H}{\partial \mfJ} &= \bbG_{11} \mfJ + \bbG_{12} \mfp \\
        \frac{\partial H}{\partial \mfp} &= \bbG_{12}^+ \mfJ + \bbG_{22} \mfp.
    \end{aligned} \notag 
\end{gather}

Напомним, что производная обратной матрицы $\displaystyle \frac{d}{d\mfq} \bbM^{-1}(\mfq)$ связана с производной $\displaystyle \frac{d}{d\mfq} \bbM(\mfq)$ следующим соотношением
\begin{gather}
    \frac{d}{d\mfq} \bbM^{-1}(\mfq) = -\bbM^{-1}(\mfq) \lsq \frac{d}{d\mfq} \bbM(\mfq) \rsq \bbM^{-1}(\mfq). \notag
\end{gather}
Воспользуемся этим соотношением при дифференцировании гамильтониана в блочном виде по вектору обобщенных координат $\mfq$.
\begin{gather}
    \frac{\partial H}{\partial \mfq} = - \frac{1}{2}
    \begin{bmatrix}
        \mfJ^+ & \mfp^+ 
    \end{bmatrix}
    \bbB^{-1} \frac{\partial \bbB}{\partial \mfq} \bbB^{-1}
    \begin{bmatrix}
        \mfJ \\
        \mfp
    \end{bmatrix} \notag
\end{gather}


Рассмотрим, как будут выглядеть уравнения для производных, если переписать гамильтониан при помощи Эйлеровых углов $\EOmega$ и импульсов $\mfpe$. Блочный вектор $\displaystyle \begin{bmatrix} \mfpe \\ \mfp \end{bmatrix}$ связан с блочным вектором $\displaystyle \begin{bmatrix} \mfJ \\ \mfp \end{bmatrix}$ следующей матрицей
\begin{gather}
    \begin{bmatrix}
        \mfpe \\
        \mfp
    \end{bmatrix} =
    \begin{bmatrix}
        \bbW & \bbzero \\
        \bbzero & \bbone \\ 
    \end{bmatrix}
    \begin{bmatrix}
        \mfJ \\
        \mfp 
    \end{bmatrix}, \quad 
    \bbW = 
	\begin{bdmatrix}
		\frac{\sin \psi}{\sin \theta} & \cos \psi & - \frac{\cos \theta \sin \psi}{\sin \theta} \\
		\frac{\cos \psi}{\sin \theta} & - \sin \psi & - \frac{\cos \theta \cos \psi}{\sin \theta} \\
		0 & 0 & 1
	\end{bdmatrix}. 
    \notag
\end{gather}

Подставим это соотношение в гамильтониан
\begin{gather}
    H = \frac{1}{2} \begin{bmatrix} \mfpe^+ & \mfp^+ \end{bmatrix}
    \begin{bmatrix}
        \bbW^+ & \bbzero \\
        \bbzero & \bbone
    \end{bmatrix}
    \begin{bmatrix}
        \bbG_{11} & \bbG_{12} \\
        \bbG_{12}^+ & \bbG_{22}
    \end{bmatrix}
    \begin{bmatrix}
        \bbW & \bbzero \\
        \bbzero & \bbone
    \end{bmatrix}
    \begin{bmatrix}
        \mfpe \\
        \mfp
    \end{bmatrix}. \notag
\end{gather}

Продифференцируем гамильтониан по блочному вектору $\displaystyle \begin{bmatrix} \mfpe \\ \mfp \end{bmatrix}$:
\begin{gather}
    \begin{bdmatrix}
        \frac{\partial H}{\partial \mfpe} \\
        \frac{\partial H}{\partial \mfp}
    \end{bdmatrix} =
    \begin{bmatrix}
        \bbW^+ & \bbzero \\
        \bbzero & \bbone
    \end{bmatrix}
    \begin{bmatrix}
        \bbG_{11} & \bbG_{12} \\
        \bbG_{12}^+ & \bbG_{22}
    \end{bmatrix}
    \begin{bmatrix}
        \bbW & \bbzero \\
        \bbzero & \bbone
    \end{bmatrix}
    \begin{bmatrix}
        \mfpe \\
        \mfp
    \end{bmatrix}. \notag
\end{gather}

Производная гамильтониана по вектору обобщенных координат $\mfq$ также получается простой
\begin{gather}
    \frac{\partial H}{\partial \mfq} = -\frac{1}{2} \begin{bmatrix} \mfpe^+ & \mfp^+ \end{bmatrix} 
    \begin{bmatrix}
        \bbW^+ & \bbzero \\
        \bbzero & \bbone
    \end{bmatrix}
    \bbB^{-1} \frac{\partial \bbB}{\partial \mfq} \bbB^{-1}
    \begin{bmatrix}
        \bbW & \bbzero \\
        \bbzero & \bbone
    \end{bmatrix}
    \begin{bmatrix}
        \mfpe \\
        \mfp
    \end{bmatrix}. \notag
\end{gather}

Так как от углов Эйлера $\EOmega$ зависит только матрица $\bbW$, то производная по ним представляет собой сумму двух слагаемых: 
\begin{gather}
    \frac{\partial H}{\partial \EOmega} = 
    \frac{1}{2} \begin{bmatrix} \mfpe^+ & \mfp^+ \end{bmatrix} 
    \begin{bdmatrix}
        \frac{\partial \bbW^+}{\partial \EOmega} & \bbzero \\
        \bbzero & \bbzero
    \end{bdmatrix}
    \bbB^{-1}
    \begin{bmatrix}
        \bbW & \bbzero \\
        \bbzero & \bbone
    \end{bmatrix}
    \begin{bmatrix}
        \mfpe \\
        \mfp
    \end{bmatrix} + \notag \\
    + \frac{1}{2} \begin{bmatrix} \mfpe^+ & \mfp^+ \end{bmatrix} 
    \begin{bmatrix}
        \bbW & \bbzero \\
        \bbzero & \bbzero
    \end{bmatrix}
    \bbB^{-1}
    \begin{bdmatrix}
        \frac{\partial \bbW}{\partial \EOmega} & \bbzero \\
        \bbzero & \bbone
    \end{bdmatrix}
    \begin{bmatrix}
        \mfpe \\
        \mfp
    \end{bmatrix}. \notag
\end{gather}

Легко заметить, что слагаемые переходят друг друга при транспонировании, а т.к. они являются скалярами, то они равны. Следовательно, выражение для этой производной сводится к удвоенному слагаемому  
\begin{gather}
    \frac{\partial H}{\partial \EOmega} = 
    \begin{bmatrix} \mfpe^+ & \mfp^+ \end{bmatrix} 
    \begin{bdmatrix}
        \frac{\partial \bbW^+}{\partial \EOmega} & \bbzero \\
        \bbzero & \bbzero
    \end{bdmatrix}
    \bbB^{-1}
    \begin{bmatrix}
        \bbW & \bbzero \\
        \bbzero & \bbone
    \end{bmatrix}
    \begin{bmatrix}
        \mfpe \\
        \mfp
    \end{bmatrix}. \notag 
\end{gather}

Последняя производная не представляется в той же степени в матричном виде, как остальные производные. Двигаясь по компонентам вектора $\EOmega$ мы будем получать разные производные $\displaystyle \frac{\partial \bbW}{\partial \EOmega}$, которые являются матрицами, и после подстановки их в квадратичную форму будем получать число. То есть, для того, чтобы получить три производные по Эйлеровым углам в рамках одного вычисления, необходимо составить трехмерный тензор $\displaystyle \frac{\partial \bbW}{\partial \EOmega}$ и осуществить свертку по двум индексам в ходе вычисления по формуле выше. (С вычислительной точки зрения это не даст никаких преимуществ, потому что операции с матрицами в специализированных библиотеках оптимизированы намного лучше чем операции с тензорами.) Если раскрыть это матричное произведение используя индивидуальные матрицы, то выражение существенно упрощается
\begin{gather}
    \frac{\partial H}{\partial \EOmega} = \mfpet \frac{\partial \bbW^+}{\partial \EOmega} \lb \bbG_{11} \bbW \mfpe + \bbG_{12} \mfp \rb \notag.
\end{gather}

Заметим, что выражение в скобках может быть выражено через производную $\displaystyle \frac{\partial H}{\partial \mfpe}$
\begin{gather}
    \frac{\partial H}{\partial \mfpe} = \bbW^+ \lb \bbG_{11} \bbW \mfpe + \bbG_{12} \mfp \rb \quad \implies \quad 
    \bbV \frac{\partial H}{\partial \mfpe} = \bbG_{11} \bbW \mfpe + \bbG_{12} \mfp, \notag
\end{gather}

где
\begin{gather}
    \bbV = \lb \bbW^+ \rb^{-1} = 
    \begin{bmatrix}
        \sin \theta \sin \psi & \cos \psi & 0 \\
        \sin \theta \cos \psi & -\sin \psi & 0 \\
        \cos \theta & 0 & 1
    \end{bmatrix}. \notag
\end{gather}

Таким образом, производные гамильтониана по Эйлеровым углам могут быть найдены через производные по Эйлеровым импульсам по следующим соотношениям
\begin{gather}
    \frac{\partial H}{\partial \EOmega} = \mfpet \frac{\partial \bbW^+}{\partial \EOmega} \bbV \frac{\partial H}{\partial \mfpe}. \notag 
\end{gather}


\end{document}

