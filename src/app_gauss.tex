Разобъем задачу на две части -- фиксируем некоторое значение равномерно распределенной матрицы $\bbS$ и рассмотрим ее действие на случайный гауссов вектор $\mf{X}$ (такой подход обоснован формулой полной вероятности). Обозначим случайную величину, являющуюся результатом действия $\bbS$ на $\mf{X}$ за $\mf{Y}$.  

\begin{gather}
	\mf{Y} = \bbS \mf{X} \notag 
\end{gather}

В силу линейности мат. ожидания:
\begin{gather}
		\bbE \left[ \mf{Y} \right] = \bbS \bbE \left[ \mf{X} \right] = \mf{0} \notag
\end{gather}

Матрица ковариации исходного вектора $\mf{X}$ кратна единичной матрице за счет того, что компоненты вектора независимы и дисперсии компонент равны
\begin{gather}
	Cov \lb \mf{X} \rb =
	\begin{bmatrix}
		Var \lb \mf{X}_1 \rb & 0 & 0 \\
		0 & Var \lb \mf{X}_2 \rb & 0 \\
		0 & 0 & Var \lb \mf{X}_3 \rb 
	\end{bmatrix} =
	\begin{bmatrix}
		\sigma^2 & 0 & 0 \\
		0 & \sigma^2 & 0 \\
		0 & 0 & \sigma^2 
	\end{bmatrix} = 
	\sigma^2 \bbE . \notag
\end{gather}

Вычислим ковариацию величины $\mf{Y}$ по определению:
\begin{gather}
		Cov \lb \mf{Y} \rb = \bbE \left[ \lb \mf{Y} - \bar{\mf{Y}} \rb \lb \mf{Y} - \bar{\mf{Y}} \rb \right] = \bbE \left[ \mf{Y} \mf{Y}^\top \right] = \bbS \, Cov \lb \mf{X} \rb \bbS^\top = \sigma^2 \bbE \notag
\end{gather}

Таким образом мы показали, что мат.ожидание и дисперсия компонент при действии фиксированной ортогональной матрицы не изменятся. Заметим, что ни мат. ожидание, ни ковариация $\mf{Y}$ не зависят от распределения $\bbS$, поэтому когда мы будем "интегрировать" по $\bbS$, чтобы учесть наличие распределение по ней, то на итоговой результат для параметров $\mf{Y}$ это не повлияет.   


