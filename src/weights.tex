\section{Веса траекторий при расчете спектра поглощения}

Рассмотрим случайный вектор начальных условий $\boldsymbol{\xi} = {(\Theta, p_R, p_{\Theta},J_X, J_Y, J_Z)}$. Очевидно, что отношение частот появления начальных условий, описываемых вектором $\boldsymbol{\xi}$ при двух разных температурах $T_1$ и $T_2$ будет определяться соотношением:

\begin{gather}
\lambda = \frac{P(\boldsymbol{\xi}, T_1)}{P(\boldsymbol{\xi}, T_2)} = \frac{\exp \lb - H \lb \boldsymbol{\xi} \rb / kT_1 \rb}{ \exp \lb -H \lb \boldsymbol{\xi} \rb / kT_2 \rb} \notag 
\end{gather}

Спектральная функция вычисляется по траекториям следующим образом:
\begin{gather}
J(\omega) = \frac{1}{N} \sum_{i=1}^{N} \hat{A} \lb \mu_i(t) \rb \label{eq:J_traj} 
\end{gather}

где $N$ -- число траекторий, а $\hat A$ -- определенного вида операция, производимая для дипольного момента на каждой траектории (квадрат модуля преобразования Фурье). Тогда вполне очевидно, что для пересчета спектральной плотности на другую температуру мы должны домножить каждый член суммы в числителе (\ref{eq:J_traj}) на соответствующий коэффициент пересчета $\lambda_i$, не забыв учесть нормировку в знаменателе.
\begin{gather}
J(\omega, T_2) = \frac{\displaystyle \sum_{i = 1}^{N} \lambda_i \hat{A} \lb \mu_i(t) \rb}{\displaystyle \sum_{i = 1}^{N}(\lambda_i \cdot 1) \label{eq:J_traj_lambda}}
\end{gather}

Для наглядности продемонстируем вышесказанное на простом примере. Пусть есть две траектории, причем коэффициент $\lambda$ для первой оказался равен $0.5$, а для второй -- $2$
\begin{equation*}
\begin{aligned}
J(\omega, T_1) = \frac{\hat A(\mu_1(t))+\hat A(\mu_2(t))}{2} = \frac{2\hat A(\mu_1(t))+2\hat A(\mu_2(t))}{4} = \\
\frac{\hat A(\mu_1(t))+\hat A(\mu_1(t))+\hat A(\mu_2(t))+\hat A(\mu_2(t))}{4}
\end{aligned}
\end{equation*}
Теперь если для первой траектории коэффициент равен $0.5$, то это означает, что при температуре $T_2$ начальное условие, соответствующее данной траектории будет встречаться в 2 раза меньше, чем при температуре $T_1$, а это означает, что в сумме (\ref{eq:J_traj}) мы в 2 раза реже будем писать преобразование от первой траектории, то есть таких членов в сумме будет в два раза меньше. Аналогичные рассуждения применяются и для второй траектории. В итоге:
\begin{equation*}
J(\omega, T_2) = \frac{\hat A(\mu_1(t))+4\hat A(\mu_2(t))}{5} = \frac{0.5\hat A(\mu_1(t))+2\hat A(\mu_2(t))}{2.5}
\end{equation*}
что соответствует формуле (\ref{eq:J_traj_lambda}).

