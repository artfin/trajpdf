\subsection{Равномерно распределенные матрицы поворота}

Следующий алгоритм к получению равномерно распределенных матриц поворота состоит из двух шагов:
\begin{enumerate}
\item равномерно распределенный поворот вокруг оси $OZ$
\item поворот, приводящий к равномерному на сфере положению северного полюса
\end{enumerate} 

Первый шаг осуществить легко; пусть случайная величина $x_1$ равномерно распределена на отрезке $[0, 1]$, тогда матрица $R$ осуществляет равномерно распределенный поворот вокруг оси $OZ$ 
\begin{gather}
R =
\begin{bmatrix}
\cos \lb 2 \pi x_1 \rb & \sin \lb 2 \pi x_1 \rb & 0 \\
- \sin \lb 2 \pi x_1 \rb & \cos \lb 2 \pi x_1 \rb & 0 \\
0 & 0 & 1
\end{bmatrix} \label{rmatrix} 
\end{gather}

Второй шаг может быть выполнен при помощи \textit{преобразования Хаусхолдера} (Householder transform).

