\documentclass[14pt]{extarticle}

\usepackage[T1]{fontenc}
\usepackage[utf8]{inputenc}
\usepackage[russian]{babel}

% page margin
\usepackage[top=2cm, bottom=2cm, left=0.5cm, right=0.5cm]{geometry}

\usepackage{wrapfig}

% AMS packages
\usepackage{amsmath, array}
\usepackage{amssymb}
\usepackage{amsfonts}
\usepackage{amsthm}

\usepackage{graphicx}
\usepackage{rotating}

\usepackage{fancyhdr}
\pagestyle{fancy}
% modifying page layout using fancyhdr
\fancyhf{}
\renewcommand{\sectionmark}[1]{\markright{\thesection\ #1}}
\renewcommand{\subsectionmark}[1]{\markright{\thesubsection\ #1}}

\rhead{\fancyplain{}{\rightmark }}
\cfoot{\fancyplain{}{\thepage }}

\usepackage{titlesec}
\titleformat{\section}{\bfseries}{\thesection.}{1em}{}
\titleformat{\subsection}{\normalfont\itshape\bfseries}{\thesubsection.}{0.5em}{}

\newcommand{\pr}{\prime}
\newcommand{\lb}{\left(}
\newcommand{\rb}{\right)}

\usepackage{mathtools} % for dfrac!

\begin{document}

\section*{Группа вращений трехмерного пространства}

Рассмотрим все вращения трехмерного пространства вокруг фиксированной точки -- начала координат. Под произведением двух вращений $g_1$ и $g_2$ будем понимать вращение $g$, состоящее в последовательном применении сначала $g_2$ и затем $g_1$. Символически запишем это так: $g = g_1 g_2$. Нетрудно проверить, что совокупность $G$ всех вращений образует группу, т.е. что при таком определении умножения выполнены все групповые аксиомы. Единицей группы $e$, единичным вращением, является поворот на нулевой угол. \par

\section*{Описание группы вращений при помощи ортогональных матриц}

Пусть $x$ -- некоторый вектор, исходящий из начала координат, вращение $g$ переводит его в вектор $x^\pr$:
\begin{gather}
		x^\pr = g x \label{rotation_action}
\end{gather}

Рассмотрим ортогональную систему координат с центром в точке $O$, обозначим через $e_1$, $e_2$, $e_3$ единичные вектора, отложенные вдоль координатных осей. Вращение $g$ переводит эту тройку векторов в тройку других взаимно ортогональных векторов, которые будем обозначать $g_1$, $g_2$, $g_3$. Вектора $g_k$, $k =1, 2, 3$ задаются проекциями на оси $e_i$, $i = 1,2,3$; обозначим через $g_{ik} = ( g_k, e_i )$ проекцию вектора $g_k$ на $i$-ую ось. Объединим проекции в матрицу
\begin{gather}
	\begin{vmatrix}
		g_{11} & g_{12} & g_{13} \\
		g_{21} & g_{22} & g_{23} \\
		g_{31} & g_{32} & g_{33}
	\end{vmatrix}
	\label{rotation_matrix}
\end{gather}
Будем обозначать эту матрицу так же $g$ и называть ее матрицей вращения $g$. Выпишем соотношение \eqref{rotation_action} покоординатно
\begin{gather}
	x_i^\pr = \sum_{k = 1}^{3} g_{ik} x_k, \label{rotation_action_coords}
\end{gathя векторов}
где $x_k$ -- координаты вектора $x$, а $x_i^\pr$ -- координаты вектора $x^\pr$. Найдем, каким условиям должны удовлетворять числа $g_{ik}$. Так как вращение не меняет длин и углов, то оно не меняет скалярного произведения векторов. Таким образом, если $x^\pr = g x$ и $y^\pr = g y$, то
\begin{gather}
	\sum_{i = 1}^{3} x_i^\pr y_i^\pr = \sum_{k = 1}^{3} x_k y_k \label{scalar_product}
\end{gather}

Подставим в левую часть равенства \eqref{scalar_product} вместо $x_i^\pr$ и $y_i^\pr$ их выражения по формуле \eqref{rotation_action_coords}:
\begin{gather}
	\sum_{i, k, l} g_{ik} \, g_{il} \, x_k y_l = \sum_{k} x_k y_k \label{scalar_product2}
\end{gather}

Сравнивая коэффициенты при произведениях $x_k y_l$ в левой и правой частях, получаем:
\begin{gather}
	\sum_{i = 1}^{3} g_{ik} \, g_{il} = \delta_{kl}, \label{scalar_product3}
\end{gather}
где $\delta_{kl}$ -- кронекеровская дельта, определенная следующими соотношениями: $\delta_{kl} = 1$, если $ k = l$, $\delta_{kl} = 0$, если $k \neq l$. Равенство \eqref{scalar_product3} может быть записано в матричной форме:
\begin{gather}
		g^\top g = e \label{scalar_product4}
\end{gather}
или
\begin{gather}
		g^\top = g^{-1}. \label{scalar_product5}
\end{gather}
Матрицы, удовлетворяющие равенствам \eqref{scalar_product4}, \eqref{scalar_product5}, называются ортогональными матрицами. Если взять детерминант обеих частей равенства \eqref{scalar_product4}, то получим $\det \lb g^\top \rb \det \lb g \rb = 1$, т.е. $\rvert \det \lb g \rb \rvert^2 = 1$, и  
\begin{gather}
	\det \lb g \rb = \pm 1. \label{unit_determinant}
\end{gather}

Итак, группа вращений $G$ может быть реализована (представлена) как группа ортогональных матриц третьего порядка с единичным детерминантом. 

\section{Введение параметров в группу вращений}

Так как каждое вращение есть вращение вокруг некоторой оси, то оно может быть полностью определено путем задания оси вращения и задания угла поворота вокруг нее. Так, вращение может быть задано вектором $\xi = (\xi_1, \xi_2, \xi_3)$, направленным вдоль оси вращения и равным по величине углу поворота. Направление вектора будем выбирать так, чтобы угол поворота не превосходил $\pi$. Координаты векторов, описывающих всевозможные вращения, будут удовлетворять условию $\xi_1^2 + \xi_2^2 + \xi_3^2 \leqslant \pi^2$, и, значит, заполнять шар радиуса $\pi$. Ясно, что различные внутренние точки шара описывают различные вращения, а две диаметрально противоположные точки на поверхности сферы -- одно и то же вращение на угол $\pi$ (поворот на угол $\pi$ в двух противоположных направлениях приводит к одному и тому же результату). \par
\textit{Такой способ описания вращений выявляет топологическую структуру группы вращений, а именно, эта группа топологически эквивалентна шару, у которого отождествлены диаметрально противоположные точки границы.} 

Представленные выше результаты показывают, что вращение $g$ может быть описано при помощи девяти параметров, а именно элементами $g_{ik}$ матрицы вращения $g$; однако эти параметры не являются независимыми, они связаны соотношениями \eqref{scalar_product3}. Примером описания вращения при помощи независимых параметров являются углы Эйлера. \par
Пусть вращение $g$ переводит координатные оси $Ox$, $Oy$, $Oz$ в оси $Ox^\pr$, $Oy^\pr$, $Oz^\pr$. Обозначим линию пересечения плоскостей $xOy$ и $x^\pr O y^\pr$ через $Ol$ (ее принято называть \textit{линией узлов}). Придадим ей направление таким образом, чтобы наблюдатель, смотря вдоль заданного направления, видел угол между осями $Oz$ и $Oz^\pr$ (меньше $\pi$), отложенным против часовой стрелки. Это условие задает направление линии узлов во всех ситуциях, за исключением тех, в которых угол между осями $Oz$, $Oz^\pr$ равен 0 или $\pi$. \par
Обозначим через $\varphi$ угол между осью $Ox$ и линией узлов $Ol$, через $\psi$ -- угол между $Ol$ и осью $Ox^\pr$ и через $\theta$ -- между $Oz$ и $Oz^\prime$. Пусть $g_\varphi$ и $g_\psi$ обозначают вращения вокруг оси $Oz$, $g_\theta$ -- вращение вокруг оси $Ox$. \par
Вращение $g$ может быть представлено композицией $g = \widetilde{g}_\psi \widetilde{g}_\theta g_\varphi$ трех поворотов $g_\varphi$, $\widetilde{g}_\theta$, $\widetilde{g}_\psi$ вокруг осей $Oz$, $Ol$, $Oz^\pr$, соответственно. В результате вращения $g_\varphi$ ось $Ox$ совпадет с линией узлов $Ol$; ось $Oz$ перейдет в ось $Oz^\pr$ в результате вращения $\widetilde{g}_\theta$; вращение $\widetilde{g}_\psi$ переведет линию узлов $Ol$ в $Ox^\pr$ (ось $Oy$ в результате вращений $g_\varphi$ и $\widetilde{g}_\psi$ перейдет в $Oy^\pr$). \par
Повороты $\widetilde{g}_\theta$ и $\widetilde{g}_\psi$ были сделаны вокруг вспомогательных осей $Ol$ и $Oz^\pr$; представим их в виде поворотов относительно первоначальных осей $Ox$ и $Oz$. Поворот $\widetilde{g}_\theta$ является преобразованием новой системы координатных осей, полученной из первоначальной, действием $g_\varphi$, следовательно $\widetilde{g}_\theta = g_\varphi g_\theta g_\varphi^{-1}$. Аналогично, $\widetilde{g}_\psi = (\widetilde{g}_\theta g_\varphi) g_\psi (\widetilde{g}_\theta g_\varphi)^{-1}$. Подставим в выражение для композиции поворотов $g$:
\begin{gather}
	g = \widetilde{g}_\psi \widetilde{g}_\theta g_\varphi = ( \widetilde{g}_\theta g_\varphi ) g_\psi ( \widetilde{g}_\theta g_\varphi)^{-1} \widetilde{g}_\theta g_\varphi = \widetilde{g}_\theta g_\varphi g_\theta = g_\varphi g_\theta g_\psi
\end{gather}
\textit{То есть, последовательность поворотов на углы $\varphi$, $\theta$, $\psi$ вокруг вспомогательных систем осей, получаемых в результате осуществления каждого следующего поворота, эквивалентна последовательности поворотов относительно исходных осей, сделанных в обратном порядке $\psi$, $\theta$, $\varphi$.} \par 
Три угла $\varphi, \theta, \psi$ являются независимыми и полностью определяют поворот $g$. Согласно определениям, они изменяются в пределах, $0 \leqslant \varphi \leqslant 2 \pi$, $0 \leqslant \psi \leqslant 2 \pi$, $0 \leqslant \theta \leqslant \pi$. Разные наборы эйлеровых углов $(\varphi, \theta, \psi)$, взятые из этих интервалов, определяют разные повороты, за исключением случаев $\theta = 0$ и $\theta = \pi$. В этих особых случаях плоскости $xOy$ и $x^\pr O y^\pr$ совпадают, и линия их пересечения, линия узлов $Ol$, оказывается неопределена. Варьируя ориентацию линии узлов в плоскости, заключаем, что в случае $\theta = 0$ пары углов $(\varphi, \psi)$ и $(\varphi + \alpha, \psi - \alpha)$ определяют один и тот же поворот для любого $\alpha$; аналогично, в случае $\theta = \pi$ пары углов $(\varphi, \psi)$ и $(\varphi + \alpha, \psi + \alpha)$ эквивалентны для любого $\alpha$.  	
Выразим элементы матрицы поворота $g$ через углы Эйлера. Вспользуемся полученным выражением для поворота $g$ через повороты $g_\varphi$, $g_\theta$ и $g_\psi$ относительно исходной системы координат.
\begin{gather}
	g_\varphi = 
	\begin{bmatrix}
		\cos \varphi & - \sin \varphi & 0 \\
		\sin \varphi & \cos \varphi & 0 \\
		0 & 0 & 1
	\end{bmatrix}, 
	\quad 
	g_\theta =
	\begin{bmatrix}
		1 & 0 & 0 \\
		0 & \cos \theta & - \sin \theta \\
		0 & \sin \theta & \cos \theta
	\end{bmatrix},
	\quad 
	g_\psi = 
	\begin{bmatrix}
		\cos \psi & - \sin \psi & 0 \\
		\sin \psi & \cos \psi & 0 \\
		0 & 0 & 1
	\end{bmatrix} \notag
\end{gather}
\begin{gather}
	g = g_\varphi g_\theta g_\psi = 
	\begin{bmatrix}
		\cos \varphi \cos \psi - \cos \theta \sin \varphi \sin \psi & - \cos \varphi \sin \psi - \cos \theta \sin \varphi \cos \psi & \sin \varphi \sin \theta \\
		\sin \varphi \cos \psi + \cos \theta \cos \varphi \sin \psi & - \sin \varphi \sin \psi + \cos \theta \cos \varphi \cos \psi & - \cos \varphi \sin \theta \\
		\sin \psi \sin \theta & \cos \psi \sin \theta & \cos \theta
	\end{bmatrix} \notag
\end{gather}

Заметим, что замена $\lb \varphi, \theta, \psi \rb \rightarrow \lb \pi - \varphi, \theta, \pi - \psi \rb$ переводит матрицу $g$ в $g^\top = g^{-1}$. То есть, если поворот $g$ задан углами $\lb \varphi, \theta, \psi \rb$, то обратный поворот задается углами $\lb \pi - \varphi, \theta, \pi - \psi \rb$.  

\section*{Связь группы вращений с группой унитарных матриц второго порядка}

Покажем, что вращения трехмерного пространства можно описывать комплексными матрицами второго порядка. Для этого рассмотрим стереографическую проекцию сферы на плоскость -- каждой точке P сферы относится точка $\zeta$ в плоскости, лежащая на луче O$^\pr$P, исходящем из северного полюса O$^\pr$. Вращение трехмерного пространства вокруг центра сферы переводит друг в друга точки сферы и порождает тем самым некоторое преобразование в плоскости. 

\begin{wrapfigure}{R}{0.3\textwidth}
\centering
\includegraphics[width=0.25\textwidth]{pictures/stereographic.png}
\caption{\label{fig:stereographic}Стереографическая проекция}
\end{wrapfigure}

Рассмотрим сферу диаметра 1. Из подобия треугольников $\Delta$ANP и $\Delta$BN$\zeta$ получаем связь между координатами $x$, $y$, $z$ точки P сферы и координатами $\xi$, $\eta$ точки $\zeta$ плоскости:
\begin{gather}
	\xi = \frac{x}{\frac{1}{2} - z}, \qquad \eta = \frac{y}{\frac{1}{2} - z}. \notag
\end{gather}

Вводим комплексную переменную $\zeta = \xi + i \eta$:
\begin{gather}
	\zeta = \xi + i \eta = \frac{x + iy}{\frac{1}{2} - z} 
\end{gather}

Т.к. точка P принадлежит сфере единичного диаметра, то ее координаты $x$, $y$, $z$ удовлетворяют соотношению
\begin{gather}
	x^2 + y^2 + z^2 = \frac{1}{4}.
\end{gather}

Используем это соотношение при преобразовании $\zeta$:
\begin{gather}
	\zeta = \frac{x + iy}{\frac{1}{2} - z} = \frac{\lb x + iy \rb \lb x - iy \rb}{\lb \frac{1}{2} - z \rb \lb x - i y \rb} = \frac{x^2 + y^2}{\lb \frac{1}{2} - z \rb \lb x - iy \rb} = \frac{ \lb \frac{1}{2} - z \rb \lb \frac{1}{2} + z \rb}{ \lb \frac{1}{2} - z \rb \lb x - i y \rb} = \frac{ \frac{1}{2} + z }{ x - iy }.  
\end{gather}

Найдем преобразование плоскости, отвечающее вращению на угол $\varphi$ вокруг оси Oz. Имеем:
\begin{gather}
	\begin{aligned}
		x^\prime &= x \cos \varphi - y \sin \varphi \\
		y^\prime &= x \sin \varphi + y \cos \varphi \\
		z^\prime &= z
\end{aligned} \\
	\zeta^\prime = \frac{x^\prime + i y^\prime}{\frac{1}{2} - z} = \frac{ x (\cos \varphi + i \sin \varphi) + iy (\cos \varphi + i \sin \varphi)}{\frac{1}{2} - z} = \exp \lb i \varphi \rb \frac{x + iy}{\frac{1}{2} - z} = \exp \lb i \varphi \rb \zeta
\end{gather}

Т.е. вращению на угол $\varphi$ отвечает преобразование плоскости $\zeta^\prime = \exp \lb i \varphi \rb \zeta$. Рассмотрим вращение на угол $\theta$ вокруг оси Ox. Заметим, что при таком вращении выражение
\begin{gather}
	\omega = \frac{y + iz}{\frac{1}{2} - x}
\end{gather}
умножается на $\exp \lb i \theta \rb$, т.е.
\begin{gather}
	\omega^\prime = \exp \lb i \theta \rb \omega \label{omega_prime_with_omega}.
\end{gather}

Выразим $\omega$ через $\zeta$ (и соответственно $\omega^\prime$ через $\zeta^\prime$). Рассмотрим отношение:
\begin{gather}
		\frac{\omega + i}{\omega - i} = \frac{\displaystyle \frac{y + iz}{\frac{1}{2} - x} + i}{\displaystyle \frac{y + iz}{\frac{1}{2} - x} - i} = \frac{y + iz + i \lb \frac{1}{2} - x \rb}{y + iz - i \lb \frac{1}{2} - x \rb} = \frac{- \lb x + iy \rb + \lb z + \frac{1}{2} \rb}{\lb x - i y \rb + \lb z - \frac{1}{2} \rb }, \\
		x + iy = \zeta \lb \frac{1}{2} - z \rb, \quad x - iy = \lb z + \frac{1}{2} \rb \zeta^{-1} \\
		\frac{\omega + i}{\omega - i} = \frac{ \zeta \lb z - \frac{1}{2} \rb + \lb z + \frac{1}{2} \rb}{ \zeta^{-1} \lb z + \frac{1}{2} \rb + \lb z - \frac{1}{2} \rb} = \zeta \label{zeta_with_omega}
\end{gather}

Аналогично получаем
\begin{gather}
	\frac{\omega^\prime + i}{\omega^\prime - i} = \zeta^\prime. \label{zeta_prime_with_omega_prime}
\end{gather}

Выражаем $\omega$ через $\zeta$ в выражении \eqref{zeta_with_omega} и $\omega^\prime$ через $\zeta^\prime$ в выражении \eqref{zeta_prime_with_omega_prime}, подставляем выражения в соотношение \eqref{omega_prime_with_omega}, связывающее $\omega$ и $\omega^\prime$:   
\begin{gather}
	\omega = -i \frac{1 + \zeta}{1 - \zeta}, \qquad \omega^\prime = - i \frac{1 + \zeta^\prime}{1 - \zeta^\prime} \\
	\frac{\zeta^\prime + 1}{\zeta^\prime - 1} = \exp \lb i \theta \rb \frac{\zeta + 1}{\zeta - 1} 
\end{gather}

Решая это уравнение относительно $\zeta^\prime$, мы получаем преобразование, отвечающее вращения на угол $\theta$ вокруг оси Ox:
\begin{gather}
	\zeta^\prime = \frac{ \zeta \lb \exp \lb i \theta \rb + 1 \rb + \lb \exp \lb i \theta \rb - 1 \rb }{ \zeta \lb \exp \lb i \theta \rb - 1 \rb + \lb \exp \lb i \theta \rb + 1 \rb} \\ 
	\frac{ \exp \lb i \theta \rb + 1 }{ \exp \lb i \theta \rb - 1 } = \frac{ \exp \lb 2 i \theta \rb - 1 }{ \exp \lb 2 i \theta \rb - 2 \exp \lb i \theta \rb + 1 } = \frac{\exp \lb i \theta \rb - \exp \lb - i \theta \rb }{ \exp \lb i \theta \rb + \exp \lb -i \theta \rb - 2 } = -i \frac{\sin \theta}{1 - \cos \theta}  = - i \ctg \frac{\theta}{2} \notag \\
	\ctg \frac{\theta}{2} = \frac{\displaystyle \cos \frac{\theta}{2}}{\displaystyle \sin \frac{\theta}{2}} = \frac{\displaystyle 2 \sin \frac{\theta}{2} \cos \frac{\theta}{2}}{\displaystyle 2 \sin^2 \frac{\theta}{2}} = \frac{\sin \theta}{1 - \cos \theta} \notag
\end{gather}

\begin{gather}
		\zeta^\prime = \frac{\zeta \lb \exp \lb i \theta \rb + 1 \rb + \lb \exp \lb i \theta \rb - 1 \rb}{\zeta \lb \exp \lb i \theta \rb - 1 \rb + \lb \exp \lb i \theta \rb + 1 \rb} = \frac{\displaystyle \zeta \frac{\exp \lb i \theta \rb + 1}{\exp \lb i \theta \rb - 1} + 1}{\displaystyle \zeta + \frac{\exp \lb i \theta \rb + 1}{\exp \lb i \theta \rb - 1}} = \frac{\displaystyle -i \zeta \ctg \frac{\theta}{2} + 1}{\displaystyle \zeta - i \ctg \frac{\theta}{2}} \\ 
	\zeta^\prime = \frac{\displaystyle \zeta \cos \frac{\theta}{2} + i \sin \frac{\theta}{2}}{\displaystyle i \zeta \sin \frac{\theta}{2} + \cos \frac{\theta}{2}}
\end{gather}

\end{document}

