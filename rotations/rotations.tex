\documentclass[14pt]{extarticle}

\usepackage[T1]{fontenc}
\usepackage[utf8]{inputenc}
\usepackage[russian]{babel}

% page margin
\usepackage[top=2cm, bottom=2cm, left=0.5cm, right=0.5cm]{geometry}

% AMS packages
\usepackage{amsmath, array}
\usepackage{amssymb}
\usepackage{amsfonts}
\usepackage{amsthm}

\usepackage{graphicx}
\usepackage{rotating}

\usepackage{fancyhdr}
\pagestyle{fancy}
% modifying page layout using fancyhdr
\fancyhf{}
\renewcommand{\sectionmark}[1]{\markright{\thesection\ #1}}
\renewcommand{\subsectionmark}[1]{\markright{\thesubsection\ #1}}

\rhead{\fancyplain{}{\rightmark }}
\cfoot{\fancyplain{}{\thepage }}

\usepackage{titlesec}
\titleformat{\section}{\bfseries}{\thesection.}{1em}{}
\titleformat{\subsection}{\normalfont\itshape\bfseries}{\thesubsection.}{0.5em}{}

\newcommand{\pr}{\prime}
\newcommand{\lb}{\left(}
\newcommand{\rb}{\right)}

\begin{document}

\section*{Группа вращений трехмерного пространства}

Рассмотрим все вращения трехмерного пространства вокруг фиксированной точки -- начала координат. Под произведением двух вращений $g_1$ и $g_2$ будем понимать вращение $g$, состоящее в последовательном применении сначала $g_2$ и затем $g_1$. Символически запишем это так: $g = g_1 g_2$. Нетрудно проверить, что совокупность $G$ всех вращений образует группу, т.е. что при таком определении умножения выполнены все групповые аксиомы. Единицей группы $e$, единичным вращением, является поворот на нулевой угол. \par

\section*{Описание группы вращений при помощи ортогональных матриц}

Пусть $x$ -- некоторый вектор, исходящий из начала координат, вращение $g$ переводит его в вектор $x^\pr$:
\begin{gather}
		x^\pr = g x \label{rotation_action}
\end{gather}

Рассмотрим ортогональную систему координат с центром в точке $O$, обозначим через $e_1$, $e_2$, $e_3$ единичные вектора, отложенные вдоль координатных осей. Вращение $g$ переводит эту тройку векторов в тройку других взаимно ортогональных векторов, которые будем обозначать $g_1$, $g_2$, $g_3$. Вектора $g_k$, $k =1, 2, 3$ задаются проекциями на оси $e_i$, $i = 1,2,3$; обозначим через $g_{ik} = ( g_k, e_i )$ проекцию вектора $g_k$ на $i$-ую ось. Объединим проекции в матрицу
\begin{gather}
	\begin{vmatrix}
		g_{11} & g_{12} & g_{13} \\
		g_{21} & g_{22} & g_{23} \\
		g_{31} & g_{32} & g_{33}
	\end{vmatrix}
	\label{rotation_matrix}
\end{gather}
Будем обозначать эту матрицу так же $g$ и называть ее матрицей вращения $g$. Выпишем соотношение \eqref{rotation_action} покоординатно
\begin{gather}
	x_i^\pr = \sum_{k = 1}^{3} g_{ik} x_k, \label{rotation_action_coords}
\end{gathя векторов}
где $x_k$ -- координаты вектора $x$, а $x_i^\pr$ -- координаты вектора $x^\pr$. Найдем, каким условиям должны удовлетворять числа $g_{ik}$. Так как вращение не меняет длин и углов, то оно не меняет скалярного произведения векторов. Таким образом, если $x^\pr = g x$ и $y^\pr = g y$, то
\begin{gather}
	\sum_{i = 1}^{3} x_i^\pr y_i^\pr = \sum_{k = 1}^{3} x_k y_k \label{scalar_product}
\end{gather}

Подставим в левую часть равенства \eqref{scalar_product} вместо $x_i^\pr$ и $y_i^\pr$ их выражения по формуле \eqref{rotation_action_coords}:
\begin{gather}
	\sum_{i, k, l} g_{ik} \, g_{il} \, x_k y_l = \sum_{k} x_k y_k \label{scalar_product2}
\end{gather}

Сравнивая коэффициенты при произведениях $x_k y_l$ в левой и правой частях, получаем:
\begin{gather}
	\sum_{i = 1}^{3} g_{ik} \, g_{il} = \delta_{kl}, \label{scalar_product3}
\end{gather}
где $\delta_{kl}$ -- кронекеровская дельта, определенная следующими соотношениями: $\delta_{kl} = 1$, если $ k = l$, $\delta_{kl} = 0$, если $k \neq l$. Равенство \eqref{scalar_product3} может быть записано в матричной форме:
\begin{gather}
		g^\top g = e \label{scalar_product4}
\end{gather}
или
\begin{gather}
		g^\top = g^{-1}. \label{scalar_product5}
\end{gather}
Матрицы, удовлетворяющие равенствам \eqref{scalar_product4}, \eqref{scalar_product5}, называются ортогональными матрицами. Если взять детерминант обеих частей равенства \eqref{scalar_product4}, то получим $\det \lb g^\top \rb \det \lb g \rb = 1$, т.е. $\rvert \det \lb g \rb \rvert^2 = 1$, и  
\begin{gather}
	\det \lb g \rb = \pm 1. \label{unit_determinant}
\end{gather}

Итак, группа вращений $G$ может быть реализована (представлена) как группа ортогональных матриц третьего порядка с единичным детерминантом. 

\section{Введение параметров в группу вращений}

Так как каждое вращение есть вращение вокруг некоторой оси, то оно может быть полностью определено путем задания оси вращения и задания угла поворота вокруг нее. Так, вращение может быть задано вектором $\xi = (\xi_1, \xi_2, \xi_3)$, направленным вдоль оси вращения и равным по величине углу поворота. Направление вектора будем выбирать так, чтобы угол поворота не превосходил $\pi$. Координаты векторов, описывающих всевозможные вращения, будут удовлетворять условию $\xi_1^2 + \xi_2^2 + \xi_3^2 \leqslant \pi^2$, и, значит, заполнять шар радиуса $\pi$. Ясно, что различные внутренние точки шара описывают различные вращения, а две диаметрально противоположные точки на поверхности сферы -- одно и то же вращение на угол $\pi$ (поворот на угол $\pi$ в двух противоположных направлениях приводит к одному и тому же результату). \par
\textit{Такой способ описания вращений выявил топологическую структуру группы вращений, а именно, эта группа топологически эквивалентна шару, у которого отождествлены диаметрально противоположные точки границы.} 

Представленные выше результаты показывают, что вращение $g$ может быть описано при помощи девяти параметров, а именно элементами $g_{ik}$ матрицы вращения $g$; однако эти параметры не являются независимыми, они связаны соотношениями \eqref{scalar_product3}. Примером описания вращения при помощи независимых параметров являются углы Эйлера. \par
Пусть вращение $g$ переводит координатные оси $Ox$, $Oy$, $Oz$ в оси $Ox^\pr$, $Oy^\pr$, $Oz^\pr$. Обозначим линию пересечения плоскостей $xOy$ и $x^\pr O y^\pr$ через $Ol$ (ее принято называть \textit{линией узлов}). Придадим ей направление таким образом, чтобы наблюдатель, смотря вдоль заданного направления, видел угол между осями $Oz$ и $Oz^\pr$ (меньше $\pi$), отложенным против часовой стрелки. Это условие задает направление линии узлов во всех ситуциях, за исключением тех, в которых угол между осями $Oz$, $Oz^\pr$ равен 0 или $\pi$.  

		
		
\end{document}
