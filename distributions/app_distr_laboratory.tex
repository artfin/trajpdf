Воспользуемся следующими двумя выводами из теории вероятностей:
\begin{enumerate}
\item Пусть случайная величина $\xi$ распределена с плотностью $f_\xi \left( x \right)$. Тогда случайная величина $\eta = a \xi + b$ распределена с плотностью
\begin{gather}
	f_\eta (x) = \frac{1}{|a|} f_\xi \lb \frac{x - b}{a} \rb \notag
\end{gather}
\item Если две \underline{независимые} случайные величины $X$ и $Y$ распределены с плотностями $X \sim f_1(x)$ и $Y \sim f_2(x)$ соответственно, то случайна величина $Z = X + Y$ распределена с плотностью
\begin{gather}
	g(z) = \int\limits_{-\infty}^{+\infty} f_1(x) f_2(z - x) dx \notag
\end{gather}
\end{enumerate}

Т.к. вектор $\mf{r} = \mf{r}_1 - \mf{r}_2$ равен разнице радиус-векторов двух атомов $\mf{r}_1$ и $\mf{r}_2$ в лабораторной системе координат соответственно, то $\dot{\mf{r}} = \dot{\mf{r}}_1 - \dot{\mf{r}}_2$. Используя п.1 и п.2 получим распределение для компонент $\mf{r}$:
\begin{gather}
\left\{
\begin{aligned}
		\dot{\mf{r}}_{1x} &\sim f_1(x) = \sqrt{ \frac{m_1}{2 \pi k T} } \exp \lb - \frac{m_1 x^2}{2 k T} \rb \\
		- \dot{\mf{r}}_{2x} &\sim f_2(x) = \sqrt{ \frac{m_2}{2 \pi k T} } \exp \lb - \frac{m_2 x^2}{2 k T} \rb 
\end{aligned} \notag \\
\dot{\mf{r}}_x \sim \int\limits_{-\infty}^{+\infty} f_1(x) f_2(z - x) dx = \frac{ \sqrt{m_1 m_2}}{2 \pi k T} \int\limits_{-\infty}^{+\infty} \exp \lb - \frac{m_1 x^2}{2 k T} \rb \exp \lb - \frac{m_2 \lb z - x \rb^2}{2 k T} \rb dx 
\right. \label{r_distr}
\end{gather}

Отдельно рассмотрим получившийся интеграл:
\begin{gather}
		\intty \exp \lb - \frac{m_1 x^2}{2 k T} - \frac{m_2 \lb z - x \rb^2}{2 k T} \rb dx = \intty \exp \lb \frac{- \lb m_1 + m_2 \rb x^2 - m_2 z^2 + 2 m_2 z x}{2 k T} \rb dx = \notag \\
		= \intty \exp \lb - \frac{\lb \sqrt{m_1 + m_2} x - \frac{m_2}{\sqrt{m_1 + m_2}} z \rb^2}{2 k T} \rb \exp \lb - \frac{m_2 z^2 - \frac{m_2^2}{m_1 + m_2} z^2 }{2 k T} \rb dx = \notag \\
		= \left[ y = \frac{ \sqrt{m_1 + m_2} x - \frac{m_2}{\sqrt{m_1 + m_2}} z}{ \sqrt{2 k T} } \right] = \sqrt{ \frac{2 k T}{m_ 1 + m_2} } \exp \lb - \frac{m_1 m_2}{2 \lb m_1 + m_2 \rb k T} z^2 \rb \intty \exp \lb - y^2 \rb dy = \notag \\
		= \sqrt{ \frac{2 \pi k T}{m_1 + m_2} } \exp \lb - \frac{m_1 m_2}{2 \lb m_1 + m_2 \rb k T} z^2 \rb \label{int}
\end{gather}

Подставляя значение интеграла \eqref{int} в выражение для плотности распределения $\dot{\mf{r}}_x$ \eqref{r_distr}, получаем
\begin{gather}
		\dot{\mf{r}}_x \sim \frac{1}{\sqrt{2 \pi k T}} \sqrt{ \frac{m_1 m_2}{m_1 + m_2} } \exp \lb - \frac{m_1 m_2}{2 \lb m_1 + m_2 \rb k T} z^2 \rb = \sqrt{ \frac{\mu}{2 \pi k T} } \exp \lb - \frac{\mu z^2}{2 k T} \rb \notag, 
\end{gather}
где через $\mu$ была обозначена приведенная масса двухатомной системы $ \mu = \displaystyle \frac{m_1 m_2}{m_1 + m_2} $.


