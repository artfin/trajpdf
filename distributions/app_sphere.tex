Пусть у нас есть сфера радиуса $R$ с введенной на ней сферической системой координат. Пусть $\theta$ -- полярный угол, а $\phi$ -- азимутальный. Тогда элемент поверхности сферы 

\begin{gather}
dS = R \sin\theta \, d\theta \, d\phi =  -R \, d \cos\theta \, d \phi \notag
\end{gather}

Переходя к малым, но конечным интервалам
\begin{gather}
\Delta S = -R \Delta \, \lb \cos\theta \rb \, \Delta \phi \notag
\end{gather}

Чтобы равномерно распределить точки на сфере, нужно разбить сферу на равные участки поверхности и расставить в каждый элемент по точке. Для этого независимо от значений углов необходимо, чтобы интервалы $\Delta \phi$ и $\Delta \lb \cos\theta \rb$ были равными. Следовательно, равномерно распределив $\cos\theta$ и $\phi$ и пересчитав с их помощью координаты точки $\left[ x, y, z \right]$, получим равномерное на сфере распределение.

