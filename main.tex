\documentclass[12pt]{article}

\usepackage[T1]{fontenc}
\usepackage[utf8]{inputenc}
\usepackage[russian]{babel}

% page margin
\usepackage[top=2cm, bottom=2cm, left=2cm, right=2cm]{geometry}

\usepackage{graphicx}

% AMS packages
\usepackage{amsmath}
\usepackage{amssymb}
\usepackage{amsfonts}
\usepackage{amsthm}

% blackboar lettering
\usepackage{dsfont}
\usepackage{bbm}

\usepackage{fancyhdr}
\pagestyle{fancy}
% modifying page layout using fancyhdr
\fancyhf{}
\renewcommand{\sectionmark}[1]{\markright{\thesection\ #1}}
\renewcommand{\subsectionmark}[1]{\markright{\thesubsection\ #1}}

\rhead{\fancyplain{}{\rightmark }}
\cfoot{\fancyplain{}{\thepage }}

\usepackage{titlesec}

% for appendix environment
\usepackage[titletoc,toc,title,page]{appendix}

\newcommand{\lb}{\left(}
\newcommand{\rb}{\right)}

\newcommand{\mf}{\mathbf}

\usepackage{dsfont}

\newcommand{\bbS}{\mathds{S}}

\newcommand{\mL}{\mathcal{L}}
\newcommand{\mN}{\mathcal{N}}
\newcommand{\intty}{\int\limits_{-\infty}^{+\infty}}

\usepackage{dsfont}
\usepackage{bbm}

\newcommand{\bbI}{\mathds{I}}
\newcommand{\bba}{\mathbbm{a}}
\newcommand{\bbA}{\mathds{A}}

\usepackage{algorithm}
\usepackage[noend]{algpseudocode}

\titleformat{\section}{\bfseries}{\thesection.}{1em}{}
\titleformat{\subsection}{\normalfont\itshape\bfseries}{\thesubsection.}{0.5em}{}

% adding code samples
\usepackage{listings}
\usepackage{xcolor}

\lstset
{
	language=C++,
	backgroundcolor=\color{black!5},
	basicstyle=\footnotesize,
	tabsize=2,
	breaklines=true,
	numbers=left
}





\newcommand{\lb}{\left(}
\newcommand{\rb}{\right)}

\begin{document}

\section{Начальные распределения для задачи двух тел: MCMC-сэмплирование и точные формулы}

Суть алгоритма Метрополиса-Гастингса заключается в конструировании случайного блуждания под графиком функции. Случайное блуждание строится по типу марковской цепи, то есть положение каждой следующей точки зависит только от положения предыдущей, память отсутствует ), такой класс алгоритмов называют MCMC-сэмплированием ( \textit{Markov chain Monte Carlo sampling}). Для построения марковской цепи привлекается семейство пробных распределений $ q( x^\prime; x^{\lb t \rb} )$, где $x^{\lb t \rb}$ -- текущее состояние; $q$ -- некое хорошо сэмплируемое распределение, сосредоточенное в окрестности текущей точки.  


\end{document}
