\documentclass[14pt]{extarticle}

\usepackage[T1]{fontenc}
\usepackage[utf8]{inputenc}
\usepackage[russian]{babel}

% page margin
\usepackage[top=2cm, bottom=2cm, left=2cm, right=2cm]{geometry}

% AMS packages
\usepackage{amsmath}
\usepackage{amssymb}
\usepackage{amsfonts}
\usepackage{amsthm}

\usepackage{graphicx}

\usepackage{fancyhdr}
\pagestyle{fancy}
% modifying page layout using fancyhdr
\fancyhf{}
\renewcommand{\sectionmark}[1]{\markright{\thesection\ #1}}
\renewcommand{\subsectionmark}[1]{\markright{\thesubsection\ #1}}

\rhead{\fancyplain{}{\rightmark }}
\cfoot{\fancyplain{}{\thepage }}

\usepackage{titlesec}
\titleformat{\section}{\bfseries}{\thesection.}{1em}{}
\titleformat{\subsection}{\normalfont\itshape\bfseries}{\thesubsection.}{0.5em}{}

\newcommand{\mf}{\mathbf}
\newcommand{\dvr}{\dot{\mathbf{r}}}
\newcommand{\vr}{\mathbf{r}}
\newcommand{\vrho}{\boldsymbol{\rho}}
\newcommand{\dvrho}{\dot{\boldsymbol{\rho}}}

\newcommand{\lb}{\left(}
\newcommand{\rb}{\right)}

\newcommand{\mH}{\mathcal{H}}

\begin{document}

\section*{Виртуальные частицы}

На примере системы N$_2$-Ar сделаем преобразование координат, аналогичное введению приведенной массы в задаче двух тел, приводящее к возникновению виртуальных масс. Обозначим массы азотов за $m_1$, $m_2$, аргона -- за $m_3$. Радиус-векторам относительно лабораторной системы координат припишем соответстующие номера. 
\begin{gather}
		T = \frac{1}{2} m_1 \dvr_1^2 + \frac{1}{2} m_2 \dvr_2^2 + \frac{1}{2} m_3 \dvr_3^2 \label{eq1}
\end{gather}

Осуществим замену переменных $\vr_1, \vr_2 \longrightarrow \vr_{12}, \vrho_{12}$:
\begin{gather}
	\left\{
	\begin{aligned}
		\vrho_{12} &= \vr_1 - \vr_2 \\
		\vr_{12} &= \frac{m_1 \vr_1 + m_2 \vr_2}{m_1 + m_2} 
	\end{aligned}
	\right. \quad \quad \Longleftrightarrow \quad \quad 
	\left\{
	\begin{aligned}
		\vr_1 &= \vr_{12} + \frac{m_2}{m_1 + m_2} \vrho_{12} \\
		\vr_2 &= \vr_{12} - \frac{m_1}{m_1 + m_2} \vrho_{12} 
	\end{aligned}
	\right. \label{eq2}
\end{gather}

Подставляя замену $\eqref{eq2}$ в выражение кинетической энергии $\eqref{eq1}$, получаем
\begin{gather}
	T = \frac{1}{2} m_{12} \dvr_{12}^2 + \frac{1}{2} \mu_{12} \dvrho_{12}^2 + \frac{1}{2} m_3 \dvr_{3}^2, \label{eq3}
\end{gather}
где были введены обозначения
\begin{gather}
	m_{12} = m_1 + m_2, \quad \quad \frac{1}{\mu_{12}} = \frac{1}{m_1} + \frac{1}{m_2}. \notag
\end{gather}
 
Заметим, что вектор $\vrho_{12}$ направлен вдоль линейной молекулы N$_2$, а вектор $\vr_{12}$ -- к ее центру масс. \\ 
Проделаем аналогичную операцию с парой векторов $\vr_{12}, \vr_{3}$, введя переменные $\vrho_\Sigma, \vr_\Sigma$: 
\begin{gather}
	\left\{
	\begin{aligned}
		\vrho_{\Sigma} &= \vr_{12} - \vr_3 \\
		\vr_\Sigma &= \frac{m_{12} \vr_{12} + m_3 \vr_3}{m_{12} + m_3}
	\end{aligned}
	\right.	\quad \quad \Longleftrightarrow \quad \quad
	\left\{
	\begin{aligned}
		\vr_3 &= \vr_\Sigma - \frac{m_{12}}{m_{12} + m_3} \vrho_\Sigma \\
		\vr_{12} &= \vr_\Sigma + \frac{m_3}{m_{12} + m_3} \vrho_\Sigma
	\end{aligned}
	\right. \\
	T = \frac{1}{2} m_\Sigma \dvr_\Sigma^2 + \frac{1}{2} \mu_\Sigma \dvrho_\Sigma^2 + \frac{1}{2} \mu_{12} \dvrho_{12}^2, \label{eq4} 
\end{gather}

где были введены обозначения
\begin{gather}
		m_\Sigma = m_1 + m_2 + m_3, \quad \textup{(сумма масс мономеров)} \notag \\
		\frac{1}{\mu_\Sigma} = \frac{1}{m_1 + m_2} + \frac{1}{m_3}. \quad  \textup{(приведенная масса мономеров)} \notag
\end{gather}

Переместив начало системы отсчета в центр масс системы, исключаем первое слагаемое в \eqref{eq4}.
\begin{gather}
	T = \frac{1}{2} \mu_\Sigma \dvrho_\Sigma^2 + \frac{1}{2} \mu_{12} \dvrho_{12}^2. \label{eq5}
\end{gather}

Введем подвижную систему координат таким образом, чтобы линейная молекула и атом всегда находились в плоскости $XZ$, а атом лежал на оси $OZ$. Введем внутренние координаты $\mf{q} = \left\{ R, \Theta \right\}$; $R$ -- длина вектора $\vrho_\Sigma$, равная расстоянию между центром масс N$_2$ и Ar, $\Theta$ -- угол поворота N$_2$ относительно линии связи в подвижной плоскости.  
\begin{gather}
	\begin{aligned}
		\vrho_\Sigma &\rightarrow \mf{R}_1 = \left\{ 0, 0, R \right\} \\ 
		\vrho_{12} &\rightarrow \mf{R}_2 = \left\{ l \cos \Theta, 0, l \sin \Theta \right\}
	\end{aligned} \quad \quad \Longleftrightarrow \quad \quad  
	\begin{aligned}
		X_1 &= 0 \\
		Y_1 &= 0 \\
		Z_1 &= R
	\end{aligned}
	\quad \quad 
	\begin{aligned}
		X_2 &= l \sin \Theta \\
		Y_2 &= 0 \\
		Z_2 &= l \cos \Theta
	\end{aligned} \notag
\end{gather}

\end{document}

