\documentclass[14pt]{extarticle}

\usepackage[T1]{fontenc}
\usepackage[utf8]{inputenc}
\usepackage[russian]{babel}

% page margin
\usepackage[top=2cm, bottom=2cm, left=0.5cm, right=0.5cm]{geometry}

% AMS packages
\usepackage{amsmath, array}
\usepackage{amssymb}
\usepackage{amsfonts}
\usepackage{amsthm}

\usepackage{graphicx}
\usepackage{rotating}

\usepackage{fancyhdr}
\pagestyle{fancy}
% modifying page layout using fancyhdr
\fancyhf{}
\renewcommand{\sectionmark}[1]{\markright{\thesection\ #1}}
\renewcommand{\subsectionmark}[1]{\markright{\thesubsection\ #1}}

\rhead{\fancyplain{}{\rightmark }}
\cfoot{\fancyplain{}{\thepage }}

\usepackage{titlesec}
\titleformat{\section}{\bfseries}{\thesection.}{1em}{}
\titleformat{\subsection}{\normalfont\itshape\bfseries}{\thesubsection.}{0.5em}{}

\newcommand{\JacH}{\left[ \, Jac \, \right]_{\textup{ham}}}

\newcommand{\mf}{\mathbf}
\newcommand{\dvr}{\dot{\mathbf{r}}}
\newcommand{\vr}{\mathbf{r}}
\newcommand{\vrho}{\boldsymbol{\rho}}
\newcommand{\dvrho}{\dot{\boldsymbol{\rho}}}

\newcommand{\vR}{\mathbf{R}}
\newcommand{\dvR}{\dot{\mathbf{R}}}
\newcommand{\mmf}{m_{\textup{mon}_1}}
\newcommand{\mms}{m_{\textup{mon}_2}}
\newcommand{\vrmf}{\mathbf{r}_{\textup{mon}_1}}
\newcommand{\vrms}{\mathbf{r}_{\textup{mon}_2}}
\newcommand{\vRms}{\mathbf{R}_{\textup{mon}_2}}
\newcommand{\dvrmf}{\dot{\mathbf{r}}_{\textup{mon}_1}}
\newcommand{\dvrms}{\dot{\mathbf{r}}_{\textup{mon}_2}}

\newcommand{\dveulerf}{\dot{\mathbf{e}}_1}
\newcommand{\dveulers}{\dot{\mathbf{e}}_2}

\newcommand{\dR}{\dot{R}}
\newcommand{\dTheta}{\dot{\Theta}}
\newcommand{\dPhi}{\dot{\Phi}}

\newcommand{\omt}{\boldsymbol{\omega}_2}
\newcommand{\Omt}{\boldsymbol{\Omega}_2}

\newcommand{\bbVf}{\mathbb{V}_1}
\newcommand{\bbVs}{\mathbb{V}_2}
\newcommand{\bbSs}{\mathbb{S}_2}
\newcommand{\dbbSs}{\dot{\mathbb{S}}_2 \,}
\newcommand{\bbIf}{\mathbb{I}_1 \,}
\newcommand{\bbIt}{\mathbb{I}_2 \,}

\newcommand{\pes}{\mathbf{p}_2^e}

\newcommand{\If}{I_1^2}
\newcommand{\Is}{I_2^2}
\newcommand{\It}{I_3^2}

\newcommand{\Iff}{I_1^1}
\newcommand{\Iss}{I_2^1}
\newcommand{\Itt}{I_3^1}

\newcommand{\dphis}{\dot{\varphi}_2}
\newcommand{\dthetas}{\dot{\theta}_2}
\newcommand{\dpsis}{\dot{\psi}_2}

\newcommand{\lb}{\left(}
\newcommand{\rb}{\right)}
\newcommand{\lsq}{\left[}
\newcommand{\rsq}{\right]}

\newcommand{\mH}{\mathcal{H}}

\newcommand{\bbS}{\mathbb{S}}
\newcommand{\ao}{\alpha_1}
\newcommand{\bo}{\beta_1}
\newcommand{\go}{\gamma_1}
\newcommand{\at}{\alpha_2}
\newcommand{\bt}{\beta_2}
\newcommand{\gt}{\gamma_2}
\newcommand{\ath}{\alpha_3}
\newcommand{\bth}{\beta_3}
\newcommand{\gth}{\gamma_3}

\makeatletter
\def\env@dmatrix{\hskip -\arraycolsep
  \let\@ifnextchar\new@ifnextchar
  \extrarowheight=2ex
  \array{*\c@MaxMatrixCols{>{\displaystyle}c}}}

\newenvironment{dmatrix}
  {\env@dmatrix}
  {\endarray\hskip-\arraycolsep}

\newenvironment{bdmatrix}
  {\left[\env@dmatrix}
  {\endmatrix\right]}
% and other matrix environments are similar
\makeatother

\begin{document}

\section*{Метод Якоби}

На примере системы N$_2$-Ar сделаем преобразование координат, аналогичное введению приведенной массы в задаче двух тел, приводящее к возникновению виртуальных масс. Обозначим массы азотов за $m_1$, $m_2$, аргона -- за $m_3$. Радиус-векторам относительно лабораторной системы координат припишем соответстующие номера. 
\begin{gather}
		T = \frac{1}{2} m_1 \dvr_1^2 + \frac{1}{2} m_2 \dvr_2^2 + \frac{1}{2} m_3 \dvr_3^2 \label{eq1}
\end{gather}

Осуществим замену переменных $\vr_1, \vr_2 \longrightarrow \vr_{12}, \vrho_{12}$:
\begin{gather}
	\left\{
	\begin{aligned}
		\vrho_{12} &= \vr_1 - \vr_2 \\
		\vr_{12} &= \frac{m_1 \vr_1 + m_2 \vr_2}{m_1 + m_2} 
	\end{aligned}
	\right. \quad \quad \Longleftrightarrow \quad \quad 
	\left\{
	\begin{aligned}
		\vr_1 &= \vr_{12} + \frac{m_2}{m_1 + m_2} \vrho_{12} \\
		\vr_2 &= \vr_{12} - \frac{m_1}{m_1 + m_2} \vrho_{12} 
	\end{aligned}
	\right. \label{eq2}
\end{gather}

Подставляя замену $\eqref{eq2}$ в выражение кинетической энергии $\eqref{eq1}$, получаем
\begin{gather}
	T = \frac{1}{2} m_{12} \dvr_{12}^2 + \frac{1}{2} \mu_{12} \dvrho_{12}^2 + \frac{1}{2} m_3 \dvr_{3}^2, \label{eq3}
\end{gather}
где были введены обозначения
\begin{gather}
	m_{12} = m_1 + m_2, \quad \quad \frac{1}{\mu_{12}} = \frac{1}{m_1} + \frac{1}{m_2}. \notag
\end{gather}
 
Заметим, что вектор $\vrho_{12}$ направлен вдоль линейной молекулы N$_2$, а вектор $\vr_{12}$ -- к ее центру масс. \\ 
Проделаем аналогичную операцию с парой векторов $\vr_{12}, \vr_{3}$, введя переменные $\vrho_\Sigma, \vr_\Sigma$: 
\begin{gather}
	\left\{
	\begin{aligned}
		\vrho_{\Sigma} &= \vr_{12} - \vr_3 \\
		\vr_\Sigma &= \frac{m_{12} \vr_{12} + m_3 \vr_3}{m_{12} + m_3}
	\end{aligned}
	\right.	\quad \quad \Longleftrightarrow \quad \quad
	\left\{
	\begin{aligned}
		\vr_3 &= \vr_\Sigma - \frac{m_{12}}{m_{12} + m_3} \vrho_\Sigma \\
		\vr_{12} &= \vr_\Sigma + \frac{m_3}{m_{12} + m_3} \vrho_\Sigma
	\end{aligned}
	\right. \\
	T = \frac{1}{2} m_\Sigma \dvr_\Sigma^2 + \frac{1}{2} \mu_\Sigma \dvrho_\Sigma^2 + \frac{1}{2} \mu_{12} \dvrho_{12}^2, \label{eq4} 
\end{gather}

где были введены обозначения
\begin{gather}
		m_\Sigma = m_1 + m_2 + m_3, \quad \textup{(сумма масс мономеров)} \notag \\
		\frac{1}{\mu_\Sigma} = \frac{1}{m_1 + m_2} + \frac{1}{m_3}. \quad  \textup{(приведенная масса мономеров)} \notag
\end{gather}

Переместив начало системы отсчета в центр масс системы, исключаем первое слагаемое в \eqref{eq4}.
\begin{gather}
	T = \frac{1}{2} \mu_\Sigma \dvrho_\Sigma^2 + \frac{1}{2} \mu_{12} \dvrho_{12}^2. \label{eq5}
\end{gather}

Это выражение может быть получено альтернативным путем. Вместо того, чтобы последовательно вводить вектора Якоби $\vrho_{12}$, $\vrho_\Sigma$, сразу выпишем выражения для них через исходные вектора $\vr_1, \vr_2, \vr_3$, дополненные выражением для вектора центра масс $\vr_\Sigma$:
\begin{gather}
	\left\{
	\begin{aligned}
			\vrho_{12} &= \vr_1 - \vr_2 \\
			\vrho_\Sigma &= \vr_{12} - \vr_3 = \frac{m_1 \vr_1 + m_2 \vr_2}{m_1 + m_2} - \vr_3 \\
			\vr_\Sigma &= \frac{m_1 \vr_1 + m_2 \vr_2 + m_3 \vr_3}{m_1 + m_2 + m_3}
	\end{aligned} 
	\right. \notag \\ 
	\begin{bmatrix}
		\vrho_{12} \\
		\vrho_\Sigma \\
		\vr_\Sigma
	\end{bmatrix}
	=
	\begin{bdmatrix}
		1 & -1 & 0 \\
		\frac{m_1}{m_1 + m_2} & \frac{m_2}{m_1 + m_2} & -1 \\
		\frac{m_1}{m_1 + m_2 + m_3} & \frac{m_2}{m_1 + m_2 + m_3} & \frac{m_3}{m_1 + m_2 + m_3} 
	\end{bdmatrix}
	\begin{bmatrix}
		\vr_1 \\
		\vr_2 \\
		\vr_3
	\end{bmatrix}
\end{gather}

Обращая выписанную матрицу, выражаем исходные вектора $\vr_1, \vr_2, \vr_3$ через вектора Якоби $\vrho_{12}, \vr_\Sigma, \vr_\Sigma$. Затем отделяем центр масс системы, то есть исключаем вектор $\vr_\Sigma$ из выражений для $\vr_1, \vr_2, \vr_3$.
\begin{gather}
	\begin{bmatrix}
		\vr_1 \\
		\vr_2 \\
		\vr_3
	\end{bmatrix}
	=
	\begin{bdmatrix}
		\frac{m_2}{m_1 + m_2} & \frac{m_3}{m_1 + m_2 + m_3} & 1 \\
		-\frac{m_1}{m_1 + m_2} & \frac{m_3}{m_1 + m_2 + m_3} & 1 \\
		0 & -\frac{m_1 + m_2}{m_1 + m_2 + m_3} & 1 
	\end{bdmatrix}
	\begin{bmatrix}
		\vrho_{12} \\ \vrho_\Sigma \\ \vr_\Sigma
	\end{bmatrix} \\
	\left\{
	\begin{aligned}
		\vr_1 &= \frac{m_2}{m_1 + m_2} \vrho_{12} + \frac{m_3}{m_1 + m_2 + m_3} \vrho_\Sigma \\
		\vr_2 &= - \frac{m_1}{m_1 + m_2} \vrho_{12} + \frac{m_3}{m_1 + m_2 + m_3} \vrho_\Sigma \\
		\vr_3 &= - \frac{m_1 + m_2}{m_1 + m_2 + m_3} \vrho_\Sigma
	\end{aligned}
	\right.
\end{gather}

Продифференцировав полученные выражения и подставив в \eqref{eq1}, приходим к \eqref{eq5} 
\begin{gather}
		T = \frac{1}{2} \mu_\Sigma \dvrho_\Sigma^2 + \frac{1}{2} \mu_{12} \dvrho_{12}^2 
\end{gather}

Введем подвижную систему координат таким образом, чтобы линейная молекула и атом всегда находились в плоскости $XZ$, а атом лежал на оси $OZ$. Введем внутренние координаты $\mf{q} = \left\{ R, \Theta \right\}$; $R$ -- длина вектора $\vrho_\Sigma$, равная расстоянию между центром масс N$_2$ и Ar, $\Theta$ -- угол поворота N$_2$ относительно линии связи в подвижной плоскости.  
\begin{gather}
	\begin{aligned}
		\vrho_\Sigma &\rightarrow \mf{R}_1 = \left\{ 0, 0, R \right\} \\ 
		\vrho_{12} &\rightarrow \mf{R}_2 = \left\{ l \cos \Theta, 0, l \sin \Theta \right\}
	\end{aligned} \quad \quad \Longleftrightarrow \quad \quad  
	\begin{aligned}
		X_1 &= 0 \\
		Y_1 &= 0 \\
		Z_1 &= R
	\end{aligned}
	\quad \quad 
	\begin{aligned}
		X_2 &= l \sin \Theta \\
		Y_2 &= 0 \\
		Z_2 &= l \cos \Theta
	\end{aligned} \notag
\end{gather}

\newpage
\section*{Гамильтониан молекулярной пары, в котором ориентация мономеров описывается относительно транслированной лабораторной системы координат}

Рассмотрим два произвольных мономера. Обозначим радиус-векторы их центров масс через $\vrmf$ и $\vrms$. Поместим начало системы отсчета в центр масс системы как целого; параметризуем ось, соединяющую центры масс мономеров, сферическими углами $\Theta, \Phi$; обозначим вектор, направленный из центра масс первого мономера в центр масс второго за $\vR$. 
\begin{gather}
	\begin{aligned}
		\vrmf &= - \frac{\mms}{M} \vR = -\frac{\mms}{M} \begin{bmatrix} R \sin \Theta \cos  \Phi \\ R \sin \Theta \sin \Phi \\ R \cos \Theta \end{bmatrix} \\
		\vrms &= \frac{\mmf}{M} \vR = \frac{\mmf}{M} \begin{bmatrix} R \sin \Theta \cos \Phi \\ R \sin \Theta \sin \Phi \\ R \cos \Theta \end{bmatrix}
	\end{aligned},  \label{com_mon}
\end{gather}
где $M$ -- сумма масс мономеров. Пусть второй мономер состоит из $n$ точек, радиус-вектора которых во введенной системе отсчета обозначим $\left\{ \vr_2^k \right\}_{k = 1 \dots n}$, а относительно центра масс второго мономера -- $\left\{ \vrms^k \right\}_{k = 1 \dots n}$. Тогда для $k$-й точки вектора связаны соотношением
\begin{gather}
	\vr_2^k = \vrms + \vrms^k 
\end{gather}

Преобразуем выражение кинетической энергии второго мономера.
\begin{gather}
		T_2 = \frac{1}{2} \sum_{k = 1}^n m_k \lb \dvr_2^k \rb^2 = \frac{1}{2} \sum_{k = 1}^n m_k \lb \dvrms + \dvrms^k \rb^2 = \frac{1}{2} \sum_{k=1}^n m_k \left[ \dvrms^2 + 2 \dvrms \dvrms^k + \lb  \dvrms^k \rb^2 \right] = \notag \\
		= \frac{1}{2} \mms \dvrms^2 + \dvrms \frac{d}{dt} \left[ \sum_{k=1}^n m_k \vrms^k \right] + \frac{1}{2} \sum_{k=1}^n m_k \lb \dvrms^k \rb^2 \label{kin1}
\end{gather}

Второе слагаемое в \eqref{kin1} равно нулю, т.к. в квадратных скобках представлен вектор, направленный в центр масс второго мономера, в системе, связанной с центром масс второго мономера, то есть нуль-вектор. Квадрат производной вектора $\vR$, соединяющего центры масс мономеров, в сферической системе координат имеет следующий вид
\begin{gather}
	\dvR^2 = \dR^2 + R^2 \dTheta^2 + R^2 \dPhi^2 \sin^2 \Theta \label{der}
\end{gather}

Подставим выражение для производной \eqref{der} в кинетическую энергию второго мономера \eqref{kin1} (вектор $\vrms$ связан с вектором $\vR$ соотношением \eqref{com_mon}):
\begin{gather}
		T_2 = \frac{1}{2} \mms \frac{\mmf^2}{M^2} \left[ \dR^2 + R^2 \dTheta^2 + R^2 \dPhi^2 \sin^2 \Theta \right] + \frac{1}{2} \sum_{k=1}^n m_k \lb \dvrms^k \rb^2 
\end{gather}

Будем описывать ориентацию второго мономера при помощи матрицы $\bbSs$. Обозначим систему векторов $\left\{ \vrms^k \right\}_{k=1 \dots n}$ в начальный момент времени за $\left\{ \vRms^k \right\}_{k=1 \dots n}$. Тогда в момент времени $t$ системы векторов связаны при помощи матрицы $\bbSs$:
\begin{gather}
	\vrms^k \lb t \rb = \bbSs \lb t \rb \vRms^k, \quad \quad \quad k = 1 \dots n
\end{gather}

Получим выражение для производной $k$-го вектора $\vrms^k$:
\begin{gather}
	\dvrms^k = \dbbSs \vRms^k = \dbbSs \bbSs^{-1} \vrms^k = \lsq \omt \times \vrms^k \rsq = \bbSs \lsq \Omt \times \vRms^k \rsq \label{der2}
\end{gather}

Просуммируем масс-взвешенные производные векторов:
\begin{gather}
\frac{1}{2} \sum_{k=1}^n m_k \lb \dvrms^k \rb^2 = \frac{1}{2} \sum_{k=1}^n m_k \lsq \Omt \times \vRms^k \right]^\top \bbSs^\top \bbSs \lsq \Omt \times \vRms^k \rsq = \frac{1}{2} \sum_{k=1}^n m_k \lsq \Omt \times \vRms^k \rsq^2 = \notag \\
	= \frac{1}{2} \sum_{k=1}^n m_k \, \Omt^\top \lsq \vRms^k \times \lsq \Omt \times \vRms^k \rsq \rsq = \frac{1}{2} \sum_{k=1}^n m_k \Omt^\top \lb \lb \vRms^k, \vRms^k \rb \Omt - \vRms^k \lb \vRms^k, \Omt \rb \rb  = \notag \\ = \frac{1}{2} \Omt^\top \bbIt \Omt, \notag
\end{gather}
где $\bbIt$ -- тензор инерции второго мономера. Положим, что в начальный момент система координат, находящаяся в центре масс пары, совпадала с системой главных осей второго мономера. Тогда тензор инерции $\bbIt$ принимает диагональный вид (верхний индекс -- номер мономера)
\begin{gather}
	\bbIt = \begin{bmatrix}
		\If & 0 & 0 \\
		0 & \Is & 0 \\
		0 & 0 & \It
	\end{bmatrix} \notag
\end{gather}

Вектор угловой скорости $\Omt$ связан с вектором эйлеровых скоростей $\dveulers$ матрицей $\bbVs$:
\begin{gather}
	\Omt = \begin{bmatrix}
		\sin \theta_2 \sin \psi_2 &  \cos \psi_2 & 0 \\
		\sin \theta_2 \cos \psi_2 & - \sin \psi_2 & 0 \\
		\cos \theta_2 & 0 & 1
	\end{bmatrix} 
	\begin{bmatrix}
		\dot{\varphi_2} \\
		\dot{\theta_2} \\
		\dot {\psi_2} 
	\end{bmatrix} =
	\bbVs \, \dveulers \notag
\end{gather}

Итак, выражение кинетической энергии второго мономера $T_2$ приходит к виду
\begin{gather}
		T_2 = \frac{1}{2} \mms \frac{\mmf^2}{M^2} \lsq \dR^2 + R^2 \dTheta^2 + R^2 \dPhi^2 \sin^2 \Theta \rsq + \frac{1}{2} \dveulers^\top \bbVs^\top \bbIt \bbVs \dveulers
\end{gather}

Проводя аналогичные рассуждения приходим к выражению для кинетической энергии первого мономера $T_1$. Получаем выражение для полной кинетической энергии пары:
\begin{gather}
		T = T_1 + T_2 = \frac{1}{2} \lsq \mmf \frac{\mms^2}{M^2} + \mms \frac{\mmf^2}{M^2} \rsq \times \lsq \dR^2 + R^2 \dTheta^2 + R^2 \dPhi^2 \sin^2 \Theta \rsq + \notag \\ + \frac{1}{2} \dveulerf^\top \bbVf^\top \bbIf \bbVf \dveulerf + \frac{1}{2} \dveulers^\top \bbVs^\top \bbIt \bbVs \dveulers,  \\
		T = \frac{1}{2} \mu \lsq \dR^2 + R^2 \dTheta^2 + R^2 \dPhi^2 \sin^2 \Theta \rsq + \frac{1}{2} \dveulerf^\top \bbVf^\top \bbIf \bbVf \dveulerf + \frac{1}{2} \dveulers^\top \bbVs^\top \bbIt \bbVs \dveulers, \label{total}
\end{gather}

где $\displaystyle \mu = \frac{\mmf \mms}{\mmf + \mms}$.

Для перехода к гамильтоновой форме кинетической энергии выпишем выражения для эйлеровых импульсов второго волчка (нижний индекс обозначает номер волчка):
\begin{gather}
	\pes = \frac{\partial T}{\partial \dveulers} = \bbVs^\top \bbIt \bbVs \dveulers \notag \\
	\begin{aligned}
		p_2^\varphi = &\If \lb \dphis \sin \theta_2 \sin \psi_2 + \dthetas \cos \psi_2 \rb \sin \theta_2 \sin \psi_2 + \\
		+ &\Is \lb \dphis \sin \theta_2 \cos \psi_2 - \dthetas \sin \psi_2 \rb \sin \theta_2 \cos \psi_2 + \\
		+ &\It \lb \dphis \cos \theta_2 + \dpsis \rb \cos \theta_2 \\
		p_2^\theta = &\If \lb \dphis \sin \theta_2 \sin \psi_2 + \dthetas \cos \psi_2 \rb \cos \psi_2 - \\
		- &\Is \lb \dphis \sin \theta_2 \cos \psi_2 - \dthetas \sin \psi_2 \rb \sin \psi_2 \\
		p_2^\psi  = &\It \lb \dphis \cos \theta_2 + \dpsis \rb
	\end{aligned}
\end{gather}

Перепишем систему в матричном виде и разрешим ее относительно вектора эйлеровых скоростей $\dveulers$:
\begin{gather}
	\begin{bdmatrix}
		\lb \If \sin^2 \psi_2 + \Is \cos^2 \psi_2 \rb \sin^2 \theta_2 + \It \cos^2 \theta_2 & \lb \If - \Is \rb \sin \theta_2 \sin \psi_2 \cos \psi_2 & \It \cos \theta_2 \\
		\lb \If - \Is \rb \sin \theta_2 \sin \psi_2 \cos \psi_2 & \If \cos^2 \psi_2 + \Is \sin^2 \psi_2 & 0 \\
		\It \cos \theta_2 & 0 & \It 
	\end{bdmatrix}
	\dveulers = \pes \notag \\
	\dveulers = \frac{1}{\If \Is \sin^2 \theta_2} \times
	\begin{bdmatrix}
		\alpha & \beta & - \alpha \cos \theta_2 \\
		\beta & \lb \If \sin^2 \psi_2 + \Is \cos^2 \psi_2 \rb \sin^2 \theta_2 & - \beta \cos \theta_2 \\
		- \alpha \cos \theta_2 & - \beta \cos \theta_2 & \frac{\If \Is}{\It} \sin^2 \theta_2 + \alpha \cos^2 \theta_2 
	\end{bdmatrix} \pes, \notag \\
	\alpha = \If \cos^2 \psi_2 + \Is \sin^2 \psi_2, \quad \beta = \lb \Is - \If \rb \sin \theta_2 \sin \psi_2 \cos \psi_2 \notag
\end{gather}

Подставив полученные выражения эйлеровых скоростей через эйлеровы импульсы в матричное произведение $\displaystyle \dveulers^\top \bbVs^\top \bbIt \bbVs \dveulers$, получаем (?) угловую часть кинетической энергии второго мономера в гамильтоновой форме
\begin{gather}
		\widetilde{T}_2^\mathcal{H} (\varphi_2, \theta_2, \psi_2, p_2^\varphi, p_2^\theta, p_2^\psi)= \frac{1}{2 \If \sin^2 \theta_2} \lsq \lb p_2^\varphi - p_2^\psi \cos \theta_2 \rb \cos \psi_2 - p_2^\theta \sin \theta_2 \sin \psi_2 \rsq^2 + \notag \\
	+ \frac{1}{2 \Is \sin^2 \theta_2} \lsq \lb p_2^\varphi - p_2^\psi \cos \theta_2 \rb \sin \psi_2 + p_2^\theta \sin \theta_2 \cos \psi_2 \rsq^2 + \frac{1}{2 \It} \lb p_2^\psi \rb^2.
\end{gather}

Выражения для импульсов, сопряженных координатам $R, \Theta, \Phi$, существенно проще:
\begin{gather}
	p_R = \frac{\partial T}{\partial \dR} = \mu \dR \notag \\
	p_\Theta = \frac{\partial T}{\partial \dTheta} = \mu R^2 \dTheta \notag \\
	p_\Phi = \frac{\partial T}{\partial \dPhi} = \mu R^2 \dPhi \sin^2 \Theta \notag
\end{gather}

Итого, приходим к следующему выражению для гамильтониана:
\begin{gather}
	T_\mathcal{H} = \frac{p_R^2}{2 \mu} + \frac{p_\Theta^2}{2 \mu R^2} + \frac{p_\Phi^2}{2 \mu R^2 \sin^2 \Theta} + \notag \\
	+ \frac{1}{2 \If \sin^2 \theta_2} \lsq \lb p_2^\varphi - p_2^\psi \cos \theta_2 \rb \cos \psi_2 - p_2^\theta \sin \theta_2 \sin \psi_2 \rsq^2 + \notag \\
	+ \frac{1}{2 \Is \sin^2 \theta_2} \lsq \lb p_2^\varphi - p_2^\psi \cos \theta_2 \rb \sin \psi_2 + p_2^\theta \sin \theta_2 \cos \psi_2 \rsq^2 + \frac{1}{2 \It} \lb p_2^\psi \rb^2 + \notag \\
	+ \frac{1}{2 \Iff \sin^2 \theta_1} \lsq \lb p_1^\varphi - p_1^\psi \cos \theta_1 \rb \cos \psi_1 - p_1^\theta \sin \theta_1 \sin \psi_1 \rsq^2 + \notag \\
	+ \frac{1}{2 \Iss \sin^2 \theta_1} \lsq \lb p_1^\varphi - p_1^\psi \cos \theta_1 \rb \sin \psi_1 + p_1^\theta \sin \theta_1 \cos \psi_1 \rsq^2 + \frac{1}{2 \Itt} \lb p_1^\psi \rb^2 \notag
\end{gather}

Рассмотрим якобиан замены переменных, приводящей гамильтониан к сумме квадратов. 
\begin{gather}
	T_\mH(p_R, p_\Theta, p_\Phi, p_1^\varphi, p_1^\theta, p_1^\psi, p_2^\varphi, p_2^\theta, p_2^\psi) \longrightarrow T_\mH(x_1, \dots, x_9) = x_1^2 + \dots + x_9^2 \\
	\begin{aligned}
		x_1 &= \frac{1}{\sqrt{2 \mu}} p_R \\
		x_2 &= \frac{1}{\sqrt{2 \mu R^2}} p_\Theta \\
		x_3 &= \frac{1}{\sqrt{2 \mu R^2 \sin^2 \Theta}} p_\Phi \\
		x_4 &= \frac{(p_1^\varphi - p_1^\psi \cos \theta_1) \cos \psi_1 - p_1^\theta \sin \theta_1 \sin \psi_1}{\sqrt{2 I_1^1 \sin \theta_1}} \\
		x_5 &= \frac{(p_1^\varphi - p_1^\psi \cos \theta_1) \sin \psi_1 + p_1^\theta \sin \theta_1 \cos \psi_1}{\sqrt{2 I_2^1 \sin^2 \theta_1}} \\
		x_6 &= \frac{1}{\sqrt{2 I_3^1}} p_1^\psi \\
		x_7 &= \frac{(p_2^\varphi - p_2^\psi \cos \theta_2) \cos \psi_2 - p_2^\theta \sin \theta_2 \sin \psi_2}{\sqrt{2 I_1^2 \sin \theta_2}} \\
		x_8 &= \frac{(p_2^\varphi - p_2^\psi \cos \theta_2) \sin \psi_2 + p_2^\theta \sin \theta_2 \cos \psi_2}{\sqrt{2 I_2^2 \sin^2 \theta_2}} \\
		x_9 &= \frac{1}{\sqrt{2 I_3^2}} p_2^\psi
	\end{aligned} 
\end{gather}

\small
\begin{sideways}
\begin{minipage}{\textheight}
\begin{gather}
	\JacH = \Bigg{|} \frac{\partial (p_R, p_\Theta, p_\Phi, p_1^\varphi, p_1^\theta, p_1^\psi, p_2^\varphi, p_2^\theta, p_2^\psi)}{\partial (x_1, x_2, x_3, x_4, x_5, x_6, x_7, x_8, x_9)} \Bigg{|} = \notag \\
	= \begin{bdmatrix}
		\frac{1}{\sqrt{2 \mu}} & 0 & 0 & 0 & 0 & 0 & 0 & 0 & 0 \\
		0 & \frac{1}{\sqrt{2 \mu R^2}} & 0 & 0 & 0 & 0 & 0 & 0 & 0 \\
		0 & 0 & \frac{1}{\sqrt{2 \mu R^2 \sin \Theta}} & 0 & 0 & 0 & 0 & 0 & 0 \\
		0 & 0 & 0 & \frac{\cos \psi_1}{\sqrt{2 I_1^1 \sin^2 \theta_1}} & -\frac{\sin \theta_1 \sin \psi_1}{\sqrt{2 I_1^1 \sin^2 \theta_1}} & - \frac{\cos \theta_1 \cos \psi_1}{\sqrt{2 I_1^1 \sin^2 \theta_1}} & 0 & 0 & 0 \\
		0 & 0 & 0 & \frac{\sin \psi_1}{\sqrt{2 I_2^1 \sin^2 \theta_1}} & \frac{\sin \theta_1 \cos \psi_1}{\sqrt{2 I_2^1 \sin^2 \theta_1}} & - \frac{\cos \theta_1 \sin \psi_1}{\sqrt{2 I_2^1 \sin^2 \theta_1}} & 0 & 0 & 0 \\
		0 & 0 & 0 & 0 & 0 & \frac{1}{\sqrt{2 I_3^1}} & 0 & 0 & 0 \\
		0 & 0 & 0 & 0 & 0 & 0 & \frac{\cos \psi_2}{\sqrt{2 I_1^2 \sin^2 \theta_2}} & - \frac{\sin \theta_2 \sin \psi_2}{\sqrt{2 I_1^2 \sin^2 \theta_2}} & - \frac{\cos \theta_2 \cos \psi_2}{\sqrt{2 I_1^2 \sin^2 \theta_2}} \\
		0 & 0 & 0 & 0 & 0 & 0 & \frac{\sin \psi_2}{\sqrt{2 I_2^2 \sin^2 \theta_2}} & \frac{\sin \theta_2 \cos \psi_2}{\sqrt{2 I_2^2 \sin^2 \theta_2}} & - \frac{\cos \theta_2 \sin \psi_2}{\sqrt{2 I_2^2 \sin^2 \theta_2}} \\
		0 & 0 & 0 & 0 & 0 & 0 & 0 & 0 & \frac{1}{\sqrt{2 I_3^2}}
	\end{bdmatrix}
\end{gather}
\end{minipage}
\end{sideways}
\normalsize

\begin{gather}
 \JacH = 2^{9/2} \mu R^2 \sqrt{I_1^1 I_2^1 I_3^1 I_1^2 I_2^2 I_3^2} \sin \Theta \sin \theta_1 \sin \theta_2 
\end{gather}

\newpage
\section*{Определение углов Эйлера вращения, являющегося композицией двух вращений}

Рассмотрим композицию двух вращений $\bbS_1$ и $\bbS_2$, параметризованновых наборами углов Эйлера $(\ao, \bo, \go)$ и $(\at, \bt, \gt)$, соответственно, определенными в Голдстейне (zxz; 313 extrinsic). 
\begin{gather}
	\bbS_1(\ao, \bo, \go) \cdot \bbS_2(\at, \bt, \gt) = \bbS_3(\ath, \bth, \gth) 
\end{gather}

Перейдем от представления вращений при помощи эйлеровых углов к кватернионному, которое позволяет более удобным образом описывать параметры вращения, являющегося композицией вращений. Рассчитав компоненты кватерниона, соотвествующего композиции вращений, через наборы углов Эйлера 1 и 2, выразим через них углы Эйлера результирующего вращения. \par
Компоненты кватерниона следующим образом связаны с углом вращения $\omega$ вокруг оси, заданной направляющими углами $\alpha$, $\beta$, $\gamma$ (с положительными направлениями осей $x$, $y$, $z$, соответственно):
\begin{gather}
	q = \begin{bmatrix} q_1 \\ q_2 \\ q_3 \\ q_4 \end{bmatrix} = 
	\begin{bdmatrix}
		\cos \frac{\omega}{2} \\
		\cos \alpha \sin \frac{\omega}{2} \\
		\cos \beta \sin \frac{\omega}{2} \\
		\cos \gamma \sin \frac{\omega}{2} 
	\end{bdmatrix}
\end{gather}

Заметим, что записанный кватернион имеет единичную норму.
\begin{gather}
	| q |^2 = q_1^2 + q_2^2 + q_3^2 + q_4^2 = \sin^2 \frac{\omega}{2} ( \cos^2 \alpha + \cos^2 \beta + \cos^2 \gamma ) + \cos^2 \frac{\omega}{2} = 1 
\end{gather}

Компоненты матрицы вращения, соотвествующей кватерниону $q = [ q_1, q_2, q_3, q_4 ]$, связаны следующим образом с компонентами кватерниона.
\begin{gather}
	M = \begin{bdmatrix}
		q_1^2 + q_2^2 - q_3^2 - q_4^2 & 2(q_2 q_3 - q_1 q_4) & 2 ( q_2 q_4 + q_1 q_3 ) \\
		2 (q_2 q_3 + q_1 q_4) & q_1^2 - q_2^2 + q_3^2 - q_4^2 & 2(q_3 q_4 - q_1 q_2) \\
		2 (q_2 q_4 - q_1 q_3) & 2 (q_3 q_4 + q_1 q_2 ) & q_1^2 - q_2^2 - q_3^2 + q_4^2
	\end{bdmatrix}
\end{gather}

Приравнивая матрицы вращения, компоненты которых выражены через углы Эйлера и через компоненты кватерниона, получают выражения для компонент кватерниона через углы Эйлера, \cite{henderson}. (При этом матрица с углами Эйлера для получения этих выражений берется обратной по Голдстейну, судя по всему это не важно.)
\begin{gather}
	M = \begin{bdmatrix}
		\cos \psi \cos \varphi - \cos \theta \sin \varphi \sin \psi & - \sin \psi \cos \varphi - \cos \theta \sin \varphi \cos \psi & \sin \theta \sin \varphi \\
		\cos \psi \sin \varphi + \cos \theta \cos \varphi \sin \psi & - \sin \psi \sin \varphi + \cos \theta \cos \varphi \cos \psi & - \sin \theta \cos \varphi \\
		\sin \theta \sin \psi & \sin \theta \cos \psi & \cos \theta
	\end{bdmatrix} \\
	q_1 = \begin{bdmatrix}
		\cos \frac{\alpha_1 + \gamma_1}{2} \cos \frac{\beta_1}{2} \\
		\cos \frac{\alpha_1 - \gamma_1}{2} \sin \frac{\beta_1}{2} \\
		\sin \frac{\alpha_1 - \gamma_1}{2} \sin \frac{\beta_1}{2} \\
		\sin \frac{\alpha_1 + \gamma_1}{2} \cos \frac{\beta_1}{2}
	\end{bdmatrix}, \quad \quad 
	q_2 = \begin{bdmatrix}
		\cos \frac{\alpha_2 + \gamma_2}{2} \cos \frac{\beta_2}{2} \\
		\cos \frac{\alpha_2 - \gamma_2}{2} \sin \frac{\beta_2}{2} \\
		\sin \frac{\alpha_2 - \gamma_2}{2} \sin \frac{\beta_2}{2} \\
		\sin \frac{\alpha_2 + \gamma_2}{2} \cos \frac{\beta_2}{2}
	\end{bdmatrix}
\end{gather}

Кватернионы представляют в виде пары $[$действительное число $q_1$, вектор $\mathbf{q} = [q_2, q_3, q_4]$ $]$: $q_1 = \lsq q_1^1, \mathbf{q}_1 \rsq$ (нижний индекс -- номер кватерниона, верхний индекс -- номер компоненты). Произведение кватернионов в векторной форме представлено соотношением
\begin{gather}
	q_1 \cdot q_2 = \lb q_1^1 q_2^1 - \mathbf{q}_1 \mathbf{q}_2 \rb + q_1^1 \mathbf{q}_2 + q_2^1 \mathbf{q}_1 + \lsq \mathbf{q}_1 \times \mathbf{q}_2 \rsq \label{prod}
\end{gather}

Подставив выражения для $q_1$ и $q_2$ в определение $\eqref{prod}$, получаем выражение для компонент кватерниона $q_3 = q_1 \cdot q_2$:
\begin{gather}
	q_3 =
	\begin{bdmatrix}
		\cos \frac{\beta_1}{2} \cos \frac{\beta_2}{2} \cos \lb \frac{\alpha_1 + \gamma_1}{2} + \frac{\alpha_2 + \gamma_2}{2} \rb - \sin \frac{\beta_1}{2} \sin \frac{\beta_2}{2} \cos \lb \frac{\alpha_1 - \gamma_1}{2} - \frac{\alpha_2 - \gamma_2}{2} \rb \\
		\sin \frac{\beta_1}{2} \cos \frac{\beta_2}{2} \cos \lb \frac{\alpha_1 - \gamma_1}{2} - \frac{\alpha_2 + \gamma_2}{2} \rb + \cos \frac{\beta_1}{2} \sin \frac{\beta_2}{2} \cos \lb \frac{\alpha_1 + \gamma_1}{2} + \frac{\alpha_2 - \gamma_2}{2} \rb \\
		\cos \frac{\beta_1}{2} \sin \frac{\beta_2}{2} \sin \lb \frac{\alpha_1 + \gamma_1}{2} + \frac{\alpha_2 - \gamma_2}{2} \rb + \sin \frac{\beta_1}{2} \cos \frac{\beta_2}{2} \sin \lb \frac{\alpha_1 - \gamma_1}{2} - \frac{\alpha_2 + \gamma_2}{2} \rb \\
			\cos \frac{\beta_1}{2} \cos \frac{\beta_2}{2} \sin \lb \frac{\alpha_1 + \gamma_1}{2} + \frac{\alpha_2 + \gamma_2}{2} \rb + \sin \frac{\beta_1}{2} \sin \frac{\beta_2}{2} \sin \lb \frac{\alpha_2 - \gamma_2}{2} - \frac{\alpha_1 - \gamma_1}{2} \rb 
	\end{bdmatrix} \label{quat}
\end{gather}

Исходя из равенства матриц вращения, компоненты которых выражены через углы Эйлера и через компоненты кватерниона, получим связь углов Эйлера с компонентами кватерниона. 
\begin{gather}
	\tg \psi = \tg \gamma_3 = \frac{M_{31}}{M_{32}} = \frac{q_2 q_4 - q_1 q_3}{q_3 q_4 + q_1 q_2} \label{eq_psi} \\
	\tg \varphi = \tg \alpha_3 = \frac{M_{13}}{M_{23}} = \frac{q_2 q_4 + q_1 q_3}{q_1 q_2 - q_3 q_4} \label{eq_varphi} \\
	\cos \theta = \cos \beta_3 = q_1^2 - q_2^2 - q_3^2 + q_4^2 \label{eq_theta} 
\end{gather}

Подстановка компонентов кватерниона \eqref{quat} в выражения для углов Эйлера \eqref{eq_psi}, \eqref{eq_varphi}, \eqref{eq_theta} была выполнена в Maple, затем выражения были автоматически упрощены. В итоге были получены следующие выражения:
\begin{gather}
	\tg \alpha_3 = \frac{\Big[ \sin (\alpha_2 + \gamma_1) \cos \alpha_1 + \cos \beta_1 \sin \alpha_1 \cos (\alpha_2 + \gamma_1) \Big] \sin \beta_2 + \sin \alpha_1 \sin \beta_1 \cos \beta_2}{\Big[ \cos \beta_1 \cos (\alpha_2 + \gamma_1) \cos \alpha_1 - \sin \alpha_1 \sin(\alpha_2 + \gamma_1) \Big] \sin \beta_2 + \cos \alpha_1 \sin \beta_1 \cos \beta_2} \\
	\cos \beta_3 = - \sin \beta_2 \cos (\alpha_2 + \gamma_1 ) \sin \beta_1 + \cos \beta_1 \cos \beta_2 \\
	\tg \gamma_3 = \frac{\Big[ \sin (\alpha_2 + \gamma_1) \cos \gamma_2 + \cos \beta_2 \sin \gamma_2 \cos (\alpha_2 + \gamma_1) \Big] \sin \beta_1 + \cos \beta_1 \sin \gamma_2 \sin \beta_2}{\Big[ \cos \beta_2 \cos (\alpha_2 + \gamma_1) \cos \gamma_2 - \sin \gamma_2 \sin (\alpha_2 + \gamma_1) \Big] \sin \beta_1 + \cos \beta_1 \cos \gamma_2 \sin \beta_2} 
\end{gather}

\begin{thebibliography}{1}
\bibitem{henderson} 
D. M. Henderson. \textit{Shuttle program: Euler angles, quaternions, and transformation matrices}.	NASA Johnson Space Center; Houston, TX, United States, 1977.

\bibitem{richter}
Dr. H. Richter, Department of Mechanical Engineering Cleveland State University. Lecture handouts. Rigid motions and homogeneous transformations.
\end{thebibliography}


\end{document}

